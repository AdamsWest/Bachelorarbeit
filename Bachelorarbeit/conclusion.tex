\chapter{Zusammenfassung und Ausblick}

\section{Zusammenfassung}
Zusammengenommen gibt es viele Einflüsse auf das Flugleistungsverhalten von elektrisch, propellergetriebenen Fluggeräten. 
Im Rahmen der Genauigkeit des Programms erweist sich ein Multicopter als geeigneter für einen Steigflug auf \SI{10000}{m} oder sogar \SI{15000}{m} Höhe. In diesem Zusammenhang sollte das Flächenflugzeug nicht vernachlässigt werden. Wie in Abschn. \ref{subsubsec:anz_mot_flaechenflzg} beschrieben wurde, ist der günstigste Flugzustand für ein Flächenflugzeug mit mehr als einem Motor der Vertikalflug. Dies legt die Verwendung eines VTOL-Fluggerätes nahe. Ein derartiges Fluggerät verbindet die Vorteile eines Multicopters, die da wären: Senkrechtstarterfähigkeiten (und Landung)und kleine Abmaße, mit denen des Flächenflugzeugs, was hauptsächlich den antriebslosen Gleitflug beinhaltet. Ein möglicher Flug sähe dann den vertikalen Steigflug bis zum Erreichen der vorgesehenen Dienstgipfelhöhe vor und einen anschließenden Sinklug im Gleiten. \\
Ein sehr großen Einfluss auf die Flugleistungen hat die Motor-Propeller-Kombination (Kap. \ref{subsubsec:mot_prop_kombi}. Prinzipiell ist ein Motor mit hohem \ensuremath{K_V}-Wert zu verwenden, der zusätzlich mit einen hohen, dauerhaften Motorstrom belastet werden kann. Der Propeller ist in Bezug auf den Motor anzupassen. \\
Die Verlustleistungen kann durch eine aerodynamisch günstige Verkleidung aller Fluggeräteinheiten minimiert werden (Kap. \ref{subsec:widerstandseinfluss}. Dies würde auch die Verstellvorrichtungen für einen Verstellpropeller oder ein Getriebe betreffen. Den mitunter signifikantesten auf die erreichbare Höhe hat die Wahl der Batterie (Kap. \ref{subsec:einfluss_n_bat} sowie \ref{sec:massenverteilung}). Als ertes sollte diese eine hohe Kapazität besitzen, da dies häufig das Kriterium für erreichbare Höhe darstellt. Weiterhin ist eine hohe Batteriespannung durch die Batteriezellenanzahl zu erreichen. Dies ist jedoch mit Rücksicht auf den Motor und mit Bedacht zu wählen, da eine hohe Batteriespannung eine Reduktion der Kapazität bewirkt (Kap. \ref{subsec:einfluss_n_bat}). Auch die Batteriemasse hat Einfluss auf die Flugleistungen. Ein Verhältnis zwischen \SI{50}{\%} und \SI{60}{\%} liefert die besten Ergebnisse (Kap. \ref{sec:massenverteilung}). Die Anzahl der Propeller oder die Größe hat bei einer äquivalenten Skalierung des Systems weiterhin einen vernachlässigbar kleinen Einfluss auf die Kostenfunktion (Kap. \ref{sec:groesse}. \\
Es stellte sich heraus, dass vor allem die maximale Motordrehzahl durch die Batteriespannung die Drehzahl des Propellers und damit die Höhe begrenzt. Ein Verstellpropeller (Kap. \ref{sec:verstellprop}) erzielt einen zusätzlichen Höhengewinn. Dies ist aber nur der Fall, wenn der Verstellmechanismus und die Aktuatorik kein zusätzliches Gewicht besitzen. Das gleiche gilt für ein Getriebe. Auch hier sind nur Höhengewinne zu erreichen, wenn das Getriebe als ideal betrachtet wird, d.h. kein Eigengewicht u. keine Verluste. Dies ergibt zusammengenommen, dass sich der Einbau eines Vertellpropellers oder eines Getriebes nicht lohnt. Für größere und schwerere Multicopter fällt das Eigengewicht nicht so sehr ins Gewicht. 


\todo[inline]{morgen nochmal überarbeiten}


%In Bezug auf die Ergebnisse sind auch viele Einschränkungen vorzunehmen

%\begin{itemize}
%	\item viele Werte für den Multicopter sind reine Schätzwerte und bedürfen einer Validierung durch Messungen und Testflügen
%	\item insgesamt beinhaltet das Programm viele einfach Modelle, so z.B. das des Motors, die viele Effekte nicht berücksichtigen
%	\item dies umfasst die Strahltheorie, das Motormodell, den Regler
%	\item beim Motor werden die Verlustgrößen (\ensuremath{R_i} und \ensuremath{I_0}) als konstant angenommen und kein Einfluss der Temperatur berücksichtigt.
%	\item dies gilt auch für das Batteriemodell, auch hier keine Berücksichtigung von Temperatureinflüssen, die bei den hier angesetzten Flughöhen definitiv zu erwarten sind 
%	\item das Modell des Flächenflugzeuges ist sehr einfach gewählt und berücksichtigt nicht die realen Gegebenheiten
%	\item es wurden viele grenzen innerhalb der flugenvelloppe vernachlässigt, die mit dem einfachen Modell nicht behandelt werden können. Das sind die Auftriebsgrenze, die Festigkeitsgrenze und die Wärme u. Temperaturgrenze. Vor allem die Auftriebsgrenze kann noch eine wichtige Rolle spielen. 
%	\item Durch die oben genannten Eigenschaften des Flächenflugzeugmodells sind auch viele Widerstände vorhanden, die nicht in das Modell mit einfließen. Diese können nur durch exakte Kenntnis der aerodynamischen Gegebenheiten, welche in Flugversuchen ermittelt werden können wie dies in \cite{Ostler.2006} der fall ist, berechnet werden (Nullwiderstand, Wellenwiderstand, etc.)
%	\item diese können nur drch Kenntnis über das Flügelprofil bestimmt werden
%	\item Es ist ein sehr einfaches Luftmodell verwendet worden, dass Reynolds oder Machzahleffekte nicht berücksichtigt. 
%	\item das sind unter anderem Einstellungen, die den Throttle betreffen (Abriegelung nach oben, um das Fluggerät handbar zu halten, oder der Modus position hold, der zur Positionshaltung wieder zusätzlich Energie benötigt)
%	\item insgesamt handelt es sich hier auch um ein statisches Modell, dynamische Effekt bleiben hierbei unberücksichtigt, z.b. Ausgleich durch Böen
%	\item Optimierungen in Richtung einer Rechenleistungserhöhung können durch die Verwendung alternativer Algorithmen (z.B. divide and conquer) erzielt werden, anstatt der hier verwendeten Brute Force Methode.
%%	\item insgs sind sehr positive Grundvoraussetzungen getroffen worden, die die Umgebungsparameter betreffen. Die Winde wurden mit einer konstanten Windgeschwindigkeit von \SI{10}{m/s} mit einbezogen. Dies ist für einen Multicopter vom Vorteil, weil es einem langsamen Vorwärtsflug gleichkommt, der den Leistungsüberschuss erhöht. Durch den Vorwärtsflug wird die induzierte Leistung verringert \cite[S.329]{Wall.2015}. Schlechtere Ergebnisse sind mit größeren oder geringeren Windgeschwindigkeiten zu erwarten.
%	\item mit den im AEROMET UAV angenommenen Umgebungsbedingungen, die eine zusätzliche Nutzlast und sehr hohe, wechselnde Windstärken annehmen, liegen die zu erwartenden Leistungen nochmal um \SI{1000}{\%} unter den hier errechneten. 
	
%	\item es zeigt sich, dass es viele verschiedene Ansatzpunkte für die Optimierung eines Fluggerätes zum effizienten Aufstieg in die untere Stratosphäre gibt. 
%	\item für diese Mission erweist sich ein Multicopter als effizienter und mit Berücksichtigung der Rahmenbedingungen einfacher zu händeln. 
%	\item hier sei wieder auf das einfache Modell des Flächenlfugzeugs verwiesen
%	\item Die Gestaltung des Copters sollte mglst. aerodynamisch sein, eine große Batterie besitzen.
%	\item Alle Komponenten sollten gut aufeinander abgestimmt sein, das betrifft vor allem die Batteriemassenverteilung, die Anzahl der Propeller
%	\item für einen Multicopter ergibt sich eine optimale Anzahl von vier Rotoren.
%	\item Bei ebendiesem Fluggerät würde sich der Einbau eines Getriebes auszahlen, wenn dieses effizient und leicht gestaltet ist.
%	\item für einen reinen Quadrocopter werden die Vorteile von einem getriebe oder einem Verstellpropeller durch dessen Nachteile überkompensiert, sodass sich Einsatz nicht auszahlt sondern eher die Flugleistungen verschlechtert. Deshalb ist umso mehr auf die vorangegangenen Punkte zu achten
%	\item der Quadrocopter aus \cite{Anderson.2018} kann als die Richtung des Designs aufgefasst werden, die für die Konstruktion eines Multicopters die besten Ergebnisse liefert.
%\end{itemize}


\section{Ausblick}
Der nächste Schritt ist die Validierung der in dieser aufgestellten Arbeit. Dies sollte möglichst in praktischen Versuchen geschehen wie dies in \cite{Ostler.2007},\cite{PCUP}. Mit der fortschreitenden Entwicklung von Batterien steigt auch die Leistungsfähigkeit und die Reichweite der UAVs. Das immer größer werdende Anwendungsfeld der unbemannten Flugeräte steigt zusätzlich, was die Weiterentwicklung nochmals beschleunigt.
Die UAVs stellen eine gute Alternative zu den Wetterballonen dar. Auch wenn sie nicht ganz über die Höhe verfügen wie sie Wetterballone erreichen. Nichtsdestotrotz stellen sie eine sichere und wiederverwendbare Lösung dar.

\begin{itemize}
	\item nächster Schritt ist die validierung der Ergebnisse in Flugversuchen und Flugmessungen. 
	\item bei der bisherigen Optimierung wurden nur einzelne Parameter ausgewählt und diese Optimiert. 
	\item an dieser Stelle fehlt eine globale Optimierung, die alle Variationen aller Parameter durchrechnet und und dann das bester Ergebnis resultiert. 
	\item dies erhöht die Genauigkeit der Untersuchung von Abhängigkeiten.
	\item Ein Modell hierzu wird z.B. in \cite{Magnussen.2015} vorgeschlagen
	\item zusätzlich ist einer Verfeinerung der Modelle anzustreben. Dies betrifft insbesondere das Flächenflugzeugmodell.
	\item Für alle untersuchungen wären Datenbanken und Modelle angemessen, die eine Abhängigkeit von der masse aufweisen.
\end{itemize}