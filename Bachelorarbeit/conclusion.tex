\chapter{Zusammenfassung und Ausblick}

\section{Zusammenfassung}
Zusammengenommen gibt es viele Einflüsse auf das Flugleistungsverhalten von elektrischen, propellergetriebenen Fluggeräten. 
Im Rahmen der Genauigkeit des Programms erweist sich ein Multicopter als geeigneter für einen Steigflug auf \SI{10000}{m} oder sogar \SI{15000}{m} Höhe. In diesem Zusammenhang sollte das Flächenflugzeug nicht vernachlässigt werden. Wie in Abschn. \ref{subsec:anz_mot_flaechenflzg} beschrieben wurde, ist der günstigste Flugzustand für ein Flächenflugzeug mit mehr als einem Motor der Vertikalflug. Dies legt die Verwendung eines VTOL-Fluggerätes nahe. Ein derartiges Fluggerät verbindet die Vorteile eines Multicopters: Senkrechtstarterfähigkeiten (und Landung)und kleine Abmaße, mit denen des Flächenflugzeugs, was hauptsächlich den antriebslosen Gleitflug beinhaltet. Ein möglicher Flug sähe den vertikalen Steigflug bis zum Erreichen der vorgesehenen Dienstgipfelhöhe vor und einen anschließenden Sinklug im Gleiten. \\
Ein sehr großen Einfluss auf die Flugleistungen hat die Motor-Propeller-Kombination (Kap. \ref{subsec:mot_prop_kombi}). Prinzipiell ist ein Motor mit hohem \ensuremath{K_V}-Wert zu verwenden, der zusätzlich mit einen hohen, dauerhaften Motorstrom belastet werden kann. Der Propeller ist in Bezug auf den Motor anzupassen. \\
Die Verlustleistungen kann durch eine aerodynamisch günstige Verkleidung aller Fluggeräteinheiten minimiert werden (Kap. \ref{subsec:widerstandseinfluss}). Dies würde auch die Verstellvorrichtungen für einen Verstellpropeller oder ein Getriebe betreffen. \\
Den mitunter signifikantesten auf die erreichbare Höhe hat die Wahl der Batterie (Kap. \ref{subsec:einfluss_n_bat}) sowie \ref{sec:massenverteilung}). Als erstes sollte diese eine hohe Kapazität besitzen, da dies häufig das Kriterium für erreichbare Höhe darstellt. Des weiteren sollte die Batterie einen Anteil von 2/3 an der Gesamtmasse vereinnahmen (Vgl. Kap. \ref{sec:massenverteilung}). Die Anzahl der Batteriezellen ist auf die Motorisierung anzupassen. Eine zu hohe Anzahl an seriell angeschlossenen Zellen führt zu einer sehr hohen nominellen Batteriespannung. Dies erhöht auf der einen Seite die Dienstgipfelhöhe, welche durch die Motorisierung gegeben ist. Auf der anderen Seite muss die PWM allerdings früher reguliert werden, um einen zerstörerischen Betrieb für den Motor und anderer Komponenten des Multicopters vorzubeugen. Gerade die geringe PWM verursacht mehr Verluste im Motorregler (Vgl. Kap. \ref{subsec:einfluss_eta_pwm}). Diese erhöhen die Effizienz einer Batterie mit weniger Zellen. Folglich ist einerseits eine Batterie mit ausreichend, seriell angeschlossenen Zellen auszuwählen, andererseits muss auch der Motorregler einen hohen Wirkungsgrad besitzen.\\
Die Anzahl der Propeller oder die Größe hat bei einer äquivalenten Skalierung des Systems weiterhin einen vernachlässigbar kleinen Einfluss auf die maximal erreichbare Höhe (Kap. \ref{sec:groesse}). \\
Es stellte sich heraus, dass vor allem die maximale Motordrehzahl durch die Batteriespannung die Drehzahl des Propellers und damit die Höhe begrenzt. Ein Verstellpropeller (Kap. \ref{sec:verstellprop}) erzielt einen zusätzlichen Höhengewinn. Dies ist aber nur der Fall, wenn der Verstellmechanismus und die Aktuatorik kein zusätzliches Gewicht besitzen. Wenn die komplette Aktuatorik und Mechanik des Verstellmechanismus weniger als \SI{10}{\%} des Gesamtgewichts einnehmen, kann aus dem Verstellpropeller ein zusätzlicher Nutzen gezogen werden. Deshalb ist der Nutzen eines Verstellpropellers gegenüber dem Zusatzgewicht abzuwägen.
Ähnliches gilt für ein stufenlos verstellbares Getriebe (Vgl. Kap. \ref{sec:getriebe}). Auch hier sind nur mäßige Höhengewinne zu erreichen, wenn das Getriebe als ideal betrachtet wird, d.h. kein Eigengewicht u. keine Verluste. Jedoch sind diese selbst im Idealfall marginal. Zusammengenommen ergibt sich, dass der hypothetische Einbau eines Getriebes keine wirklichen Vorteile mit sich bringt. Für größere und schwerere Multicopter fällt das Eigengewicht weniger ins Gewicht. \\
Alle Ergebnisse fließen zusammen und bilden am Ende die bestmögliche Konfiguration (Vgl. Kap. \ref{sec:endergebnis}). Diese erreicht bereits unter idealen Umgebungsbedingungen mehr als \SI{18000}{m} Höhe. \\
Für das Projekt AEROMET\_UAV sind als letzter Punkt noch die für das Projekt festgelegten Randbedingungen von Bedeutung. Mit diesen erreicht selbst die bestmögliche Konfiguration gerade einmal \SI{12000}{m} Höhe (Vgl. Kap. \ref{sec:aeromet_rb}). Mit vorgeschriebenen Grenzen für Entladung der Batterie ist ein Aufstieg von elektrischen, propellergetriebenen Fluggerät auf \SI{10000}{m} als kritisch zu betrachtet.


\section{Ausblick}
Der nächste Schritt ist die Validierung der in dieser Arbeit aufgestellten Ergebnisse. Das umfasst insbesondere Flugversuche und Flugleistungsmessungen wie z.B. in \cite{Ostler.2006} oder \cite{PCUP.2017}.
Außerdem ist eine Verfeinerung der Modelle, die dem Programm zugrunde gelegt wurden, anzustreben. Insbesondere das Modell des Motorreglers, der Motoren oder der Batterie benötigen eine Abhängigkeit von der Temperatur.
Ebenso ist das Modell des Flächenflugzeugs zu erweitern. Bisher ist nur die Leistungsgrenze berücksichtigt worden, nicht aber z.B. die Auftriebsgrenze oder die Festigkeitsgrenze innerhalb der Flugenveloppe. 
Ein weiterer wichtiger Punkt ist, dass allen Modellen bis auf das Modell des Motors eine Massenabhängigkeit fehlt. Ein Beispiel dafür ist die Erhöhung der Gleitleistung eines Flugzeuges oder eine Verringerung der Flügeldicke. Werden diese Parameter geändert, so zieht das zusätzlich eine Änderung der Flügelform oder Profildicke nach sich, die wiederum eine verstärkte Flügelstruktur und Flugzeugzelle und letztendlich eine Anpassung der Masse erfordern. Dieser Zusammenhang findet hier keine Anwendung. Es fehlen Funktionen und Datenbanken, um diese Abhängigkeit darzustellen.\\
Bei der bisherigen Untersuchung und Optimierung von Parametern von unbemannten Fluggeräten ist bisher nur der Einfluss einzelner Parameter unter Festhaltung aller übrigen Parameter betrachtet worden. An dieser Stelle fehlt eine globale Optimierung, die alle Variationen aller Parameter durchrechnet und am Ende das beste Ergebnis aller Parameterkombinationen präsentiert. Dies erhöht die Genauigkeit und berücksichtigt einen gegenseitigen Einfluss von zu untersuchenden Aspekten. Ein Modell hierzu wird z.B. in \cite{Magnussen.2015} vorgeschlagen. 

%Der nächste Schritt ist die Validierung der in dieser Arbeit aufgestellten Ergebnisse. Dies sollte möglichst in praktischen Versuchen geschehen wie dies in \cite{Ostler.2007} oder\cite{PCUP} vollführt wird. Mit der fortschreitenden Entwicklung von Batterien steigt auch die Leistungsfähigkeit und die Reichweite der UAVs. Das immer größer werdende Anwendungsfeld der unbemannten Flugeräte steigt zusätzlich, was die Weiterentwicklung nochmals beschleunigt.
%Die UAVs stellen eine gute Alternative zu den Wetterballonen dar. Auch wenn sie nicht ganz über die Höhe verfügen wie sie Wetterballone erreichen. Nichtsdestotrotz stellen sie eine sichere und wiederverwendbare Lösung dar.

%Bei der bisherigen Untersuchung und Optimierung von Parametern von unbemannten Fluggeräten sind bisher nur der Einfluss einzelner Parameter betrachtet worden. An dieser Stelle fehlt eine globale Optimierung, die alle Variationen aller Parameter durchrechnet und am Ende das beste Ergebnis aller Parameterkombinationen präsentiert. Dies Erhöht die Genauigkeit und berücksichtigt einen gegenseitigen Einfluss von zu untersuchenden Aspekten. Ein Modell hierzu wird z.B. in \cite{Magnussen.2015} vorgeschlagen. \\
%Weiterhin ist eine Verfeinerung der Modelle anzustreben. Dies betrifft insbesondere das Flächenflugzeugmodell. Bisher wurden Massenabhängigkeiten nur bei den Motoren berücksichtigt. Dies fehlt für andere Einheiten, z.B. die Propeller, den Rahmen, die Batterie, usw. Insbesondere das Flächenflugzeugmodell bedarf weiterer Verfeinerung. Dies umfasst die Spezifizierung weitere Begrenzungen im Bereich der Flugenveloppe. 

%Interessant wären noch Funktionen oder Datenbanken, die bei einer Änderung der Flugzeug






