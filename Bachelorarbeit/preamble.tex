% ******************************************************************************
%
% Laden der Dokumentenklasse Report
\documentclass[%
    a4paper,        % Papiergroesse
    nexus,          % Schriftart [nexus,arial]
    11pt,           % Schriftgroesse
    lnum,           % Die Paket-/Klassenoption lnum wählt für das Dokument einen
                    % Schriftschnitt mit Versalziffern.
    smallchapters,  % Stellt kleinere Chapter Ueberschriften ein.
    oneside,        %
    %halfparskip,   %
    %fleqn,         % Gleichungen werden links ausgerichtet
    %style=screen,   % Für die bessere Darstellung auf dem Bildschirm. Fuer den
                    % Druck entfernen.
    %monochrome,
    rgb,
    svgnames,       % Laden von Farbnamen vordefinierter Farben aus xcolor
    parskip=full,	% Absaetze statt Einzug
]{tubsreprt}        % Definition der Dokumentenklasse
% ******************************************************************************
%
% Laden der noetigen Latex Pakete, sowie Einstellungen im Praeamble, die fuer
% das Dokument benoetigt werden. Dort wird auch das Paket mit den eigenen Makros
% geladen.
% ******************************************************************************
%
% Es ist zu beachten, dass bei einer Vielzahl an geladenen Paketen der Compiler
% folgende Fehlermeldung "! No room for a new \dimen ." ausgibt, wenn ein
% Zaehler 233 Register allokiert hat. Zuerst muss das Paket morewrites and etex
% geladen werden, um eine hoehere Anzahl Schreibregister zu ermoeglichen, da
% ansonsten nur 256 Register bereitstehen. Mit etex stehen 32768 Register zur
% Verfuegung. Siehe hierzu:
% http://tex.stackexchange.com/questions/38607/no-room-for-a-new-dimen
\usepackage{morewrites}
% see http://www.tex.ac.uk/cgi-bin/texfaq2html?label=noroom
% After update for LaTeX released after 2015 not rquired anymore. Activate for
% previous version.
% For more Informationi see:
% https://tex.stackexchange.com/questions/38607/no-room-for-a-new-dimen
%\usepackage{etex}
%\reserveinserts{28}
% ******************************************************************************
%
% Pruefung des Textes mit neuer deutscher Rechtschreibung
\usepackage[ngerman]{babel}
%
% Uebersetzt unterschiedliche Begriffe, die in LaTeX vorhanden sind bei der
% Ausgabe in das Deutsche.
% Beispielsweise: \SI{1.234}{\metre}
\usepackage[ngerman]{translator}
% ******************************************************************************
%
% Das inputenc-Paket ermoeglicht die direkte Eingabe von Sonderzeichen.
\usepackage[utf8]{inputenc}
% ******************************************************************************
%
% Biber Paket fuer die Literaturliste
%\usepackage{multibib}
\usepackage[backend=biber]{biblatex}
%\usepackage[
%    backend=biber,
%    style=authoryear-icomp,
%    sortlocale=de_DE,
%    natbib=true,
%    url=false,
%    doi=true,
%    eprint=false,
%    bibencoding=utf8,safeinputenc=true
%]{biblatex}
\addbibresource{Literatur.bib}
% ******************************************************************************
%
% Textsatz
%
% Das microtype-Paket bringt optischen Randausgleich und minimale Skalierung der
% Buchstaben. Diese »font-expansion« verbessert den Zeilenumbruch, reduziert
% Trennstellen und erhoeht den Grauwert der Seite. Es werden Wortzwischenraeume
% im Blocksatz gleichmaessiger. Die Zeit eines bei der Erstellung der tex
% Dokumentes verlaengert sich, da die Schriftart-Varianten berechnet werden
% muessen. Seit 2007 kann das Paket sich auch automatisch um die leichte
% Sperrung von Kapitaelchen kuemmern. Mit diesem Paket wird die Anzahl von
% "under-" und "overfull box" Warnungen innerhalb der Kompilierungs Logdateien
% verringert.
\usepackage{microtype}
%
% Das winzige Paket ellipsis kuemmert sich um den Leerraum rund um die
% Auslassungspunkte. Es kann bedenkenlos immer geladen werden.
\usepackage{ellipsis}
%
% Schusterjungen und Witwenregelung
% Verhindert einzelne erste Zeilen unten
\clubpenalty = 10000
% Verhindert einzelne letzte Zeilen oben
\widowpenalty = 10000
\displaywidowpenalty = 10000
%
% Dieses Paket hilft die Zeilenabstaende dynamisch anzupassen. Eine andere Art
% um Zeilenabstaende im Text zu definieren. Mit \onehalfspacing wird
% Zeilenabstand auf 1.5 gesetzt.
%\usepackage{setspace}
%
% Mit dem \parskip Befehl und einer Laengenangabe ist es moeglich die Groesse
% eines Absatzes zu bestimmen.
%\usepackage{parskip}
% ******************************************************************************
%
% Formatierungen
%
% Erscheinungsbild der Bild- und Tabellenbeschriftung anpassen
\usepackage[hang,small,bf]{caption}
%
% Entfernt das Einbinden der ToC (Table of Contents)
\usepackage[nottoc]{tocbibind}
% ******************************************************************************
%
% Pakete, die fuer die Darstellung mathematischer Gleichungen notwendig sind
\usepackage[tbtags]{amsmath}
\usepackage{amsfonts}
\usepackage{amssymb}
\usepackage{mathtools}
%
\usepackage{commath}
%
\usepackage{bm}
%
% Streichen von Termen zu einem Wert, beispielsweise ->0
%\usepackage[
%    smaller,
%    %makeroom
%]{cancel}
%
% Ermoeglicht die fleqn Umgebung, um einzelne Gleichungen linksbuendig
% darzustellen.
% Aufgrund von einem zuvor definierten \nr Makro, steht es im Konflikt mit
% dem Paket nccmath.
%\let\oldnr=\nr
%\let\nr\undefined
%
%\usepackage{nccmath}
%
% Erlaube Zerilenumbrueche in Formeln
%\allowdisplaybreaks
%
% Erzeuge einen Operator Rang
%\DeclareMathOperator{\Rang}{Rang}
% Vektorpfeil mit \vv{} darstellen
%\usepackage{esvect}
%\MakeRobust{\vv}
% ******************************************************************************
%
% Paket, um Werte und Einheiten darzustellen und Tabellen mit Ziffern zu
% Formatieren.
\usepackage{siunitx}
%
% Einstellungen fuer eine deutsche Ausgabe
\sisetup{output-decimal-marker = {,}}
\sisetup{locale = DE}
\sisetup{per-mode = symbol}
%
% Laden von Abkuerzungen
\sisetup{load-configurations = abbreviations}
%\sisetup{load=prefixed}
% ******************************************************************************
%
% Sonderzeichen
%
% Erlaubt dei Verwendung von Sonderzeichen. Importiert beispielsweise das
% Promill Zeichen.
%\usepackage{textcomp}
%
% Dieses Paket laedt erweiterte Symbole
%\usepackage{latexsym}
%
% Weitere Sonderzeichen und Symbole
%\usepackage{pifont}
%
% @ wird als Buchstabe interpretiert
%\makeatletter
% ******************************************************************************
%
% Allgemein benoetigte Pakete
%
% Internetadressen direkt verlinken. Einfuegen von URLs mit:
% \url{http://www.Seitennahme.de/}
%\usepackage{url}
%
% Erlaubt Unterstreichen mit \uline, \uuline \uwave usw.
%\usepackage{ulem}
%EXAMPLE UNDERLINE
%\underline{\smash{The quick brown fox jumped over the lazy dog.}}
%\setul{5pt}{.4pt}% 5pt below contents
%\ul{The quick brown fox jumped over the lazy dog.} \par
%\setul{1pt}{.4pt}% 1pt below contents
%\ul{The quick brown fox jumped over the lazy dog.} \par
%
% Dieses Paket wird benoetigt, um stellenweise Querformat nutzen zu koennen.
% Genutzt wird das Querformat mit:
% \begin{landscape} Text \end{landscape}.
%\usepackage{lscape}
%
% Um Tabellen auf die breite der Seite zu bekommen, werden mit "X" markierte
% Spalten gestaucht, die anderen nicht
%\usepackage{tabularx}
%
% Erlaubt in Tabellen die Zusammenfassung mehrerer Spalten einer Zeile zu einer.
% Fuer die zusammengafsste Spalte ist im Gegensatz zu \multirow{}{}{} kein
% Leerfeld mittels "&" zu setzten.
%\usepackage{array,multicol}
%
% Aufzaehlungen
%\usepackage{enumitem}
%
% drehen Tab,Fig:\begin{sideways},\begin{rotate}{30}
\usepackage{rotating}
%\usepackage{hvfloat}
% ******************************************************************************
%
% Erweiterung zu color mit Zugriff auf verschiedene Arten
%\usepackage[svgnames,table]{xcolor}
% ******************************************************************************
%
% Allgemeine Darstellungen von Grafiken
%
% Wird benoetigt fuer die Darstellung von Grafiken.
\usepackage{graphicx}
%
% Stelle die Pfade fuer Grafiken zur Verfuegung
\graphicspath{{images/}{images/}}
%
% Mit diesem Paket und dem Befehl '\begin{figure}[H]' bzw. der Position [H],
% koennen Bilder genau an die Stelle im Text gesetzt werden, wo das Bild
% eingefuegt wird.
\usepackage{float}
%
% Ermoeglicht das darstellen von vielen (unter) Bildern als ein Bild.
%\usepackage{subfigure}
\usepackage{subfig}
% ******************************************************************************
%
% Ermoeglicht das Einbinden von PDF Dateien.
\usepackage{pdfpages}
% ******************************************************************************
%
% Dieses Paket ermoeglicht PGF plots innerhalb der TIKZ Umgebung
\usepackage{pgf}
%
% Ermoeglicht eine Darstellung von (vielen) Werte in einem Diagramm mit
% unterschiedlichen Eigenschaften und vielen Funktionen.
\usepackage{pgfplots}
%
% Wahl der PGF Version.
\pgfplotsset{compat=newest}
%
%\usepgfplotslibrary{external}
%
% Einstellen des Gleitzahltrennzeichens in PGF plots als Komma (deutsch)
\pgfplotsset{/pgf/number format/set decimal separator={,}}
%
\pgfplotsset{
    every axis/.append style={
        %line width=1pt,
        grid style={line width=1.0pt},
        tick style={line width=1.0pt, black}
    }
}
%
% Ermoeglicht den Zugriff auf das Paket \usepackage{siunitx} und die Verwendung
% von Einheiten innerhalb PGF Plots.
\usepgfplotslibrary{units}
%\usepgfplotslibrary{external}
%\tikzexternalize
%
% Spezieller Befehl, der eine Transformation von XY-Koordinaten des PGF
% Koordinatensystems in das Plot Koordinatensystem ermoeglicht.
%\makeatletter
%    \newcommand\transformxdimension[1]{
%        \pgfmathparse{%
%            ((#1/\pgfplots@x@veclength)+\pgfplots@data@scale@trafo@SHIFT@x)/%
%        10^\pgfplots@data@scale@trafo@EXPONENT@x}
%    }
%    \newcommand\transformydimension[1]{
%        \pgfmathparse{%
%            ((#1/\pgfplots@y@veclength)+\pgfplots@data@scale@trafo@SHIFT@y)/%
%       10^\pgfplots@data@scale@trafo@EXPONENT@y}
%    }
%\makeatother
% ******************************************************************************
%
% Tikz ermoeglicht das erstellen von Grafiken und Funktionsgraphen
% unterschiedlicher Art.
%
% Laden des Tikz Bild Format Paketes mit entsprechenden Bibliotheken.
\usepackage{tikz}
\usetikzlibrary{% Laden von unterschiedlichen TIKZ Bibliotheken.
    %shapes,
    %backgrounds,
    %shadows,
    %arrows,
    %matrix,
    %calc,
    positioning,
    %decorations.pathreplacing,
    %snakes,
%    intersections,
    %through,
    %patterns,
%    decorations.markings,
    %spy,                            % Ermoeglicht ein Zoomen in TIKZ Bildern
    %3d,
    %quotes,
    %angles,
    plotmarks
}
\tikzset{every picture/.append style={font=\small}}
\tikzset{
    every picture/.append style={
        line width=1.0pt,
        %thick,
        mark size=3pt
    }
}
%
% Erstelle die Tikz Bilder separat
% Nutze für pdflatex:  -shell-escape
%\usetikzlibrary{external}
%\tikzexternalize[prefix=figures/externalized/,shell escape=-enable-write18]
%\tikzset{external/system call={pdflatex
%     \tikzexternalcheckshellescape -halt-on-error -synctex=1
%     -file-line-error-style --shell-escape -extra-mem-top=20000000
%     -output-directory=build -interaction=batchmode -jobname "\image" "\texsource"}
%     }
%\tikzexternalize[prefix=fig_externalized/,
%	optimize=true, optimize command away=\includepdf,
%	%up to date check=diff,
%	up to date check=md5,
%]
%\tikzset{external/force remake}
%\tikzexternalize
%  *****************************************************************************
% Custom legend
% argument #1: any options
%\makeatletter
%\newenvironment{customlegend}[1][]{%
%    \begingroup
%    % inits/clears the lists (which might be populated from previous
%    % axes):
%    \pgfplots@init@cleared@structures
%    \pgfplotsset{#1}%
%}{%
%    % draws the legend:
%    \pgfplots@createlegend
%    \endgroup
%}%
%
%% makes \addlegendimage available (typically only available within an
%% axis environment):
%\def\addlegendimage{\pgfplots@addlegendimage}
%\makeatother
%  *****************************************************************************
%
% Werkzeuge zum Zeichnen euklidischer Geometrie.
%\usepackage{tkz-euclide}
%\usetkzobj{all}
%
% Tikz spezifisch Einstellungen
%
% Spy on plots
%\tikzset{%
%    new spy style/.style={%
%        spy scope={%
%            magnification=5,
%            size=1.25cm,
%            connect spies,
%            every spy on node/.style={%
%                rectangle,
%                draw,
%            },
%            every spy in node/.style={%
%                draw,
%                rectangle,
%                fill=gray!40,
%            }
%        }
%    }
%}
%
% Get X and Y out of an point in TIKZ
%\makeatletter
%\newcommand{\gettikzxy}[3]{%
%    \tikz@scan@one@point\pgfutil@firstofone#1\relax
%    \edef#2{\the\pgf@x}%
%    \edef#3{\the\pgf@y}%
%}
%
% ******************************************************************************
% Some default plotting lengths for figure sizes
%\makeatletter
%\newcommand{\figureheight}{%
%    \pagegoal \advance-\pagetotal
%}
%\def\@figureheight{
%    \dimexpr\pagegoal
    %\addtolength{\figureheight}{-\pagetotal}
    %\addtolength{\figureheight}{-\footskip}
%}
%\makeatother
\newlength\figurewidth
%\setlength{\figurewidth}{0.75\textwidth}
\setlength{\figurewidth}{0.9\textwidth}
\newlength\figureheigth
\setlength{\figureheigth}{0.75\textheight}
%
% twofigurewidth besitzt die laenge 0.5*(\textwidth-90pt)
% 90pt wird genutzt, da der pgf Plot zwei Axen mit jeweils einer Beschriftung
% haben kann.
\newlength\twofigurewidth
\setlength{\twofigurewidth}{\textwidth}
\addtolength{\twofigurewidth}{-45pt}
\setlength{\twofigurewidth}{0.5\twofigurewidth}
\newlength\twofigureheight
\setlength{\twofigureheight}{\textheight}
%\addtolength{\twofigureheight}{-45pt}
\setlength{\twofigureheight}{0.5\twofigureheight}
%
% twofigurewidth besitzt die laenge 0.5*(\textwidth-90pt)
% 90pt wird genutzt, da der pgf Plot zwei Axen mit jeweils einer Beschriftung
% haben kann.
\newlength\twofigurewidthxx
\setlength{\twofigurewidthxx}{\textwidth}
\addtolength{\twofigurewidthxx}{-45pt}
\setlength{\twofigurewidthxx}{0.5\twofigurewidthxx}
\newlength\twofigureheightxx
\setlength{\twofigureheightxx}{\textheight}
\addtolength{\twofigureheightxx}{-45pt}
\setlength{\twofigureheightxx}{0.5\twofigureheightxx}
%
% twofigurewidth besitzt die laenge 0.8*\twofigurewidth
\newlength\twofigurewidths
\setlength{\twofigurewidths}{0.80\twofigurewidth}
\newlength\twofigureheights
\setlength{\twofigureheight}{0.80\twofigureheight}
\newlength\threefigureheights
\setlength{\threefigureheights}{0.75\twofigureheight}
% ******************************************************************************
%
% Verlinkungen innerhalb des Dokumentes
%
\usepackage[
    breaklinks=true,
    colorlinks=false,
    %frenchlinks=false,
    %bookmarksnumbered=true
]{hyperref}                 % Package fuer Lesezeichen und Verlinkungen
% Beschreibung der Parameter
% breaklinks=boolean        : Gibt an, ob Links umgebrochen werden duerfen
% colorlinks=boolean        : Links eingefaerbt
% linkcolor=color           : Farbe der Dokument-internen Links
% citecolor=color           : Farbe der Links zum Literaturverzeichnis
% filecolor=color           : Farbe der Links auf lokale Dateien
% urlcolor=color            : Farbe der Links zu externe URLs
% frenchlinks=booelean      : Links werden als smallcaps, anstatt farbig
%                             dargestellt
% bookmarksnumbered=boolean : Kapitelnummern werden im Inhaltsverzeichnis
%                             angezeigt
\hypersetup{
    pdftitle    = {Berechnung von Pi},
    pdfsubject  = {Um was geht es \ldots},
    pdfauthor   = {Max Mustermann},
    pdfkeywords = {Rocket, Aerothermodynamics, Aerodynamics, flow Control},
    pdfcreator  = {pdflatex},
    pdfproducer = {LaTeX with hyperref}
}
% ******************************************************************************
%
% Einbinden von Tabellen
%
%\usepackage{csvsimple}
%
% Ermoeglicht das Erstellen von Tabellen ueber mehrere Seiten.
\usepackage{longtable}
%
% Erstellen von Tabellen mit PGF
%\usepackage{pgfplotstable}
%
% Ermoeglicht den Import von *.csv Tabellen
%\usepackage{csvtools}
% Use ; seperators
%\setcsvseparator{,}
% Use tab seperators
%\setcsvseparator{^^I}
%
% Zusammenfassung mehrerer Reihen in einer Tabellenspalte
%\usepackage{multirow}
%
% Zusammenfassung mehrerer Spalten in einer Tabellenreihe
%\usepackage{multicolumn}
%
% Erzeugt hochwertigere horizontale Striche in Tabellen
%\usepackage{booktabs}
% ******************************************************************************
%
% Einstellungen fuer die Darstellung von Quelltexten
%
% ******************************************************************************
%
% Dieses Paket stellt die Befehle fuer die Gesamtzahl der Seiten, Abbildungen
% und Tabellen zur Verfuegung.
\usepackage[figure,table]{totalcount}
\usepackage{lastpage}
% ******************************************************************************
%
% Einbinden von eigenen Makros
%
\usepackage{my_macropackage}
% ******************************************************************************
%
% Ermoeglicht das erstellen von zu erledigenden TODO Notizen.
%
% Bei Fertigstellung sollte dieses Paket AUSKOMMENTIERT werden!
\usepackage{todonotes}
% ******************************************************************************
%
% Package for fenerating dummy text/blindtext
\usepackage{blindtext}
% ******************************************************************************
%
% Deklaration von Schriftarten für Formeln
\DeclareOldFontCommand{\rm}{\normalfont\rmfamily}{\mathrm}
\DeclareOldFontCommand{\sf}{\normalfont\sffamily}{\mathsf}
\DeclareOldFontCommand{\tt}{\normalfont\ttfamily}{\mathtt}
\DeclareOldFontCommand{\bf}{\normalfont\bfseries}{\mathbf}
\DeclareOldFontCommand{\it}{\normalfont\itshape}{\mathit}
\DeclareOldFontCommand{\sl}{\normalfont\slshape}{\@nomath\sl}
\DeclareOldFontCommand{\sc}{\normalfont\scshape}{\@nomath\sc}
\DeclareRobustCommand*\cal{\@fontswitch\relax\mathcal}
\DeclareRobustCommand*\mit{\@fontswitch\relax\mathnormal}
% ******************************************************************************
%******************************************************************************
%
% Packages zum Erstellen von Struktogrammen
\usepackage{struktex}
%*******************************************************************************
% Packages zum einfachen Erstellen einer Projektdokumentierung, Workpackagedescription, Workbreakdownstructure und Gantt-Chart

\usepackage{lscape}

%
% Dieses Paket ermoeglicht PGF plots innerhalb der TIKZ Umgebung
\usepackage{pgf}
\usepackage{pgfgantt}
%
% Ermoeglicht ein viel einfacheres erstellen von WPDs
\usepackage{colortbl}
\usepackage{totcount}
\let\titleoriginal\title           % save original \title macro
\renewcommand{\title}[1]{          % substitute for a new \title
    \titleoriginal{#1}%               % define the real title
    \newcommand{\otitle}{#1}        % define \otitle
}
\usepackage{workpackagedescription}

\usepackage{tikz}

\usetikzlibrary{ shapes, shadows, arrows}
%*******************************************************************************