\documentclass{article}
\usepackage{struktex}
\usepackage{geometry}
\usepackage[ngerman]{babel}
\geometry{a5paper, top=0mm, left=-0.5mm, right=5mm, bottom=10mm}



\begin{document}

\pagestyle{empty}

\vspace*{\fill}

\begin{struktogramm}(140,200)
\assign[1]{Multicopter- und Umgebungsparameter festlegen (im Startskript)}
\assign[1]{Diskretisierungen (Geschwindigkeit, Höhe) festlegen}
\assign[1]{Aufruf des Hauptskripts: Leistungsberechnung starten}
\while[5]{Für alle Zeilen der APC-Datenbank}
	\ifthenelse[10]{1}{1}{Stimmt Durchmesser mit dem gesuchten überein?}{ja}{nein}
		\assign[2]{Gehe zur nächsten Zeile}
		\change
		\assign[2]{Lösche Zeile}
	\ifend
\whileend
\while[5]{Für alle Propeller}
	\assign[2]{Extrahiere Propellerkennfeld}
	\assign[2]{Speicher das Ergebnis unter fortlaufenden Nummern}
	\assign[2]{Erhöhe Propellerzähler}
\whileend
\assign[1]{Initialisierung der Parameterberechnung}
\while[5]{F\"ur alle Höhenabschnitte}
	\assign[1]{H\"ohe, Dichte, Luftdruck Temperatur berechnen}
	\assign[1]{arithmetische Mittelwert berechnen}
	\assign[1]{Schub- und Leistungskennfeld anpassen}
	\assign[2]{Initialisierung der Leistungsberechnung}
	\while[5]{Für alle Bahngeschwindigkeiten}
		\assign[2]{Initialisierungen}
		\while[5]{Für alle Propeller}
			\assign[2]{\texttt{\textbf{Leistungsberechnung}}}
		\whileend
		\ifthenelse[10]{4}{1}{Sind die Werte NaN?}{nein}{ja}
			\while[5]{Solange Abbruchkriterium nicht erreicht}		
				\assign[2]{Finde den Index mit der geringsten verbrauchten Energiemenge}
				\ifthenelse[10]{1}{1}{Werte innerhalb Leistungsgrenzen?}{ja}{nein}
				\assign[2]{Verlasse Schleife}
				\change
				\assign[2]{Suche nächst kleinere Energiemenge}
				\ifend
			\whileend
			\assign[2]{Übergabe aller Leistungsparameter mit diesem Index}
			\change
			\assign[2]{Verwerfe alle Ergebnisse}
		\ifend
		\assign[2]{Berechne benötigte Energie für Steiggeschwindigkeit}
	\whileend
	\ifthenelse[10]{4}{1}{Sind die Werte NaN?}{nein}{ja}
		\while[5]{Solange Abbruchkriterium nicht erreicht}		
			\assign[2]{Finde den Index mit der geringsten verbrauchten Energiemenge}
			\ifthenelse[10]{1}{1}{Werte innerhalb Leistungsgrenzen?}{ja}{nein}
			\assign[2]{Verlasse Schleife}
			\change
			\assign[2]{Suche nächst kleinere Energiemenge}
			\ifend
		\whileend
		\assign[2]{Übergabe aller Leistungsparameter mit diesem Index}
		\change
		\assign[2]{Verwerfe alle Ergebnisse}
	\ifend
	\assign[2]{Erhöhe Zählervariable}
\whileend
\assign[2]{Ergebnisse der Leistungsparameter in Diagrammen speichern}
\assign[2]{Speichern der Diagramme in .pdf-Datei}
\end{struktogramm}

\end{document}