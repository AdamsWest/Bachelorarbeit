\documentclass{article}
\usepackage{struktex}
\usepackage{geometry}
\usepackage[ngerman]{babel}
\geometry{a5paper, top=0mm, left=-0.5mm, right=5mm, bottom=10mm}



\begin{document}

\pagestyle{empty}

\vspace*{\fill}

\begin{struktogramm}(140,200)
\while[5]{Für alle Bahngeschwindigkeiten}
	\assign[2]{Initialisierungen}
	\while[5]{Für alle Propeller}
	\assign[2]{Berechne Gesamtmasse}
	\assign[2]{Flugzeit für Höhenschritt berechnen}			
	\while[5]{Solange Abbruchkriterium nicht erreicht}
		\assign{Aerodynamik berechnen}
	\whileend
	\assign[2]{Schub berechnen}
	\assign[2]{Schub auf Propeller verteilen}
	\ifthenelse[10]{1}{4}{Schub zu gro\ss{}?}{ja}{nein}
		\assign[2]{Ergebnis verwerfen (NaN)}
		\change
		
		\assign[2]{Drehzahl und Drehmoment aus Propellerkennfeld interpolieren}
		\assign[2]{Motorzustand berechnen}
		\assign[2]{Zustand der Motorregler berechnen}
		\assign[2]{Zustand der Batterie neu berechnen}
		\assign[2]{Gesamtwirkungsgrad berechnen}
	\ifend
	\ifthenelse[10]{1}{1}{Werden Grenzen überschritten?}{ja}{nein}
		\assign[2]{Ergebnis verwerfen (NaN)}
		\change
		\assign[2]{Ergebnis beibehalten}
	\ifend
	\assign[2]{Speichern der aufgebrachten Energiemenge}
	\whileend
\whileend
\ifthenelse[10]{5}{1}{Sind die Werte NaN?}{nein}{ja}
	\while[5]{Solange Abbruchkriterium nicht erreicht}		
		\assign[2]{Finde den Index mit der geringsten verbrauchten Energiemenge}
		\ifthenelse[10]{1}{3}{Alle Werte innerhalb Leistungsgrenzen?}{ja}{nein}
			\assign[2]{Verlasse Schleife}
			\change
			\assign[2]{Suche nächst kleinere Energiemenge}
		\ifend
	\whileend
	\assign[2]{Übergabe aller Leistungsparameter mit diesem Index}
		\change
	\assign[2]{Verwerfe alle Ergebnisse}
\ifend
\end{struktogramm}

\end{document}