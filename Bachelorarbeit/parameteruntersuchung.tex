\chapter{Parameteruntersuchung}
\label{chap:parameteruntersuchung}

\section{Einleitung und Vorgehensweise}
\label{sec:einleitung_und_vorgehensweise}

\section{Steiggeschwindigkeit}
\label{sec:steiggeschwindigkeit}

\begin{itemize}
	\item Abhängigkeit des Widerstandes von \ensuremath{V_{Kg}} muss gegeben sein
	\item Krit. für Auswahl optimaler Steiggeschwindigkeit notwendig, alleine min. Batteriekapazitätentnahme nicht mgl., da sonst zu geringe Steiggeschwindigkeiten herauskommen
	\item Struktogramm folgt
\end{itemize}

\section{Verstellpropeller}
\label{sec:verstellprop}

\begin{itemize}
	\item Aus der Datenbank von APC werden alle Propeller mit dem vorgegebenen Durchmesser und verschiedenen Pitches entnommen
	\item Vernachlässigt werden dabei gesonderte Propeller mit z.B. besonderem Pitch, Verwindung, E am Ende oder mit mehr als 2 Blättern (3 oder 4)
	\item \textcolor{red}{hier interessant, ob Funktion für Regression des Wirkungsgrades über dem Schubbeiwert???}
	\item interessant ist hier Regression --> Frage nur über was, Wirkungsgrad in Abhängigkeit von \ensuremath{C_T} oder \ensuremath{\lambda}
	\item Auswahl nach 
	\begin{equation}
		\eta_{Prop} = \frac{T\cdot\tilde{V}}{P} = \frac{T\cdot (\mu_z + v_i)}{M\cdot \omega}
	\end{equation}
\end{itemize}
