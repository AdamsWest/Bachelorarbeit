\chapter{Parameteruntersuchung}
\label{chap:parameteruntersuchung}

\section{Einleitung und Vorgehensweise}
\label{sec:einleitung_und_vorgehensweise}
Grundlegend kann die Parameteruntersuchung wie eine Art Entscheidungsbaum aufgefasst werden. Dabei führt jede Entscheidung im Baum zu einer neuen Untersuchung und zu neuen Erkenntnissen. Im Verlaufe dieser Untersuchung werden somit Konzepte, Flugzustände, Komponenten und Konstellationen ausgewählt und intensiver betrachtet. Den Beginn zeichnet die grundlegende Frage aus, welches Fluggerätekonzept, i.e. Flächenflugzeug oder Multicopter, sich für einen effizienten Aufstieg in die untere Stratosphäre optimaler erweist.   

\section{Multicopter im Vergleich zu einem Flächenflugzeug}
\label{sec:multicopter_vs_flaechenflugzeug}


\subsection{Vorgehensweise beim Vergleich}
\label{sec:vorgehenswiese_vergleich}
Jedes Luftfahrzeugkonzept entzieht sich einem direkten Vergleich mit einem Luftfahrzeug einer anderen Art. So weist jedes Fluggerät in seiner Gattung spezifische Vorteile auf wie der Start ohne Landebahn und das Hovern in der Luft für Multicopter oder der Gleitflug für Flächenflugzeuge. Die optimale Auslegung beider führt zu unterschiedlichen Designs was die Propeller, die Motorleistung und -gewicht, Größe, Gesamtgewicht etc. betrifft. Aus diesem Grund müssen Kriterien für eine Vergleichbarkeit vorgeschrieben werden. Hierfür wird das Design des Multicopters auf das aus \cite{Anderson.2018} festgelegt, was genauer in Kapitel \ref{sec:komponenten} beschrieben ist. Da die Flugleistungen von \cite{Anderson.2018} bekannt sind und der Quadrocopter durchaus schon im Rahmen der Anforderungen für diese Mission als optimiert betrachtet werden kann, bedarf es lediglich einer Untersuchung des Flächenflugzeuges. Dazu wird das Flächenflugzeug auf Parameter fixiert, mit denen es bereits sehr hoch aufsteigen kann. Zur Untersuchung und Vergleichbarkeit werden beide Gesamtmassen gleichgesetzt \ensuremath{m_{ges,Quadrocopter} = m_{ges,Flächenflugzeug}}. Dabei setzt sich die Masse der Flächenflugzeugbatterie   
\begin{equation}
	m_{Bat,Fl} = m_{Bat,Quad} + (m_{Mot,Quad}\cdot n_{Prop,Quad} - m_{Mot,Fl}\cdot n_{Prop,Fl}) - (1-f_P)\cdot m_{Quad}  
\end{equation}
in Bezug auf bereits gewählte Massen und auf den Quadrocopter zusammen. Der Faktor \ensuremath{f_P} kann als Penaltyfaktor verstanden werden. Dieser verringert zusätzlich die Batteriemasse, wenn das Strukturgewicht des Flächenflugzeugs das des Quadrocopters überschreitet
\begin{equation}
	f_P = \frac{m_{Flächenflugzeug}}{m_{Quad}}.
\end{equation} 
Für erste Untersuchungen kann der Penaltyfaktor auf 1 gesetzt werden. Dies entspricht einer sehr optimistischen Einschätzung. Im Anschluss werden die Parameter in näherer Umgebung der ersten festgesetzten Werte variiert. Dadurch kann der Einfluss auf das Leistungsverhalten und die Richtung der Optimierung bestimmt werden. Diese erste, einfache Untersuchung ist nur eine sehr oberflächliche, weil jeder Parameter nur einzeln untersucht wird. Jegliche Kombinationen von Einflüssen wie der Einfluss des Masse auf die Steiggeschwindigkeit oder vergleichbare Beziehungen werden vernachlässigt. Im Hinblick auf diese erste, kleine Optimierung ist der Kostenfaktor die maximal erreichbare Höhe beider Fluggeräte. Je nachdem welches der beiden Fluggeräte effektiver und effizienter eine maximale Flughöhe erreicht, wird es weiter untersucht und anschließend optimiert. 


\subsection{Erste Untersuchung}
\label{subsec:erste_untersuchung}
In der folgenden Tabelle sind wichtige Parameter des Flächenflugzeuges aufgelistet.

\begin{center}
	\captionof{table}{wichtige Parameter des Flächenflugzeugs}
	\begin{tabular}{l l l} \hline
		Parameter & Variablenname & Wert \\ \hline
		Motormasse \ensuremath{m_{Mot}}& \texttt{m\_Mot} & \SI{106}{g} \\
		Geschwindigkeitskonstante \ensuremath{K_V} & \texttt{K\_V} & \SI{1390}{1/(V\cdot s)} \\
		maximaler Dauerstrom \ensuremath{I_{max}} & \texttt{I\_max} & \SI{30}{A} \\
		Propeller & \texttt{prop\_name} & 9x7 \\
		Anzahl Propeller \ensuremath{n_{Prop}} & \texttt{n\_prop} & \SI{1}{} \\
		Auslegungsgleitzahl \ensuremath{E^{\star}} & \texttt{E\_stern} & \SI{4}{} \\
		Auslegungsgeschwindigkeit \ensuremath{V^{\star}} & \texttt{V\_stern} & \SI{100}{km/h} \\
		Gleitzahl \ensuremath{E} & \texttt{E} & \SI{4}{} \\ \hline
	\end{tabular}	
	\label{tab:flzg_parameter}
\end{center}

Die gewählte Konstellation erreicht fast \SI{11500}{m} Höhe. Der begrenzende Faktor ist in diesem Fall die fehlende Leistung zum Aufstieg in noch größere Höhen. Zu Beginn des Steigflugs stellt sich ein optimaler Bahnneigungswinkel von ca. \SI{55}{^\circ} ein. Dieser Winkel kann bis zu einer Höhe \SI{7500}{m} gehalten werden. Dabei steigt die absolute Fluggeschwindigkeit linear mit dem Produkt aus \ensuremath{\sqrt{\rho^\star/\rho}} an (Vgl. Gleichung \ref{eq:geschw_flaechenflugzeug}). Gleichzeitig steigt der Motorstrom leicht, da für einen konstanten Bahnneigungswinkel der benötigte Schub konstant bleibt (Vgl. Gleichung \ref{eq:schub_flaechenflugzeug}), aber mit abnehmenden Dichte das Drehmoment zunimmt. Noch stärker als der Motorstrom steigt die Motorspannung linear an, bis sie das Niveau des Batteriestroms erreicht. Damit ist das Verhältnis von \ensuremath{U_{Mot}} und \ensuremath{U_{Bat}} gleich 1 und  die PWM liegt bei \SI{100}{\%}. Ab diesem Zeitpunkt kann \textcolor{red}{leistungsbedingt} die Geschwindigkeit und Schub für einen konstanten Steigwinkel nicht mehr gehalten werden. Folglich steigt der Bahnneigungswinkel, da für ein Höhenintervall die absolute Fluggeschwindigkeit und analog die Steiggeschwindigkeit \ensuremath{V_H} mit einem mit einem  größeren Bahnneigungswinkel sinkt (Vgl. Gleichung \ref{eq:geschw_flaechenflugzeug}). Die maximale Motorspannung entspricht ab \SI{7500}{m} Flughöhe der maximalen Batteriespannung, die durch die Last von anfänglich \SI{15,6}{V} auf ca. \SI{14.7}{V} einbricht. Der Verlauf des Batteriestroms steht in direktem Zusammenhang mit dem Motorstrom und der Motorspannung. Dies wird aus Gleichung \ref{eq:batteriestrom} ersichtlich. Bei einem beinahe konstantem Motorstrom ist \ensuremath{I_{Bat}} fast ausschließlich von \ensuremath{U_{Mot}} abhängig. Daher der gleiche Verlauf wie bei \ensuremath{U_{Mot}}. Danach ist \ensuremath{U_{Mot}} konstant und \ensuremath{I_{Bat}} hängt nur noch von \ensuremath{I_{Mot}} ab. 
Der Verlauf der Drehzahl ist ausschlaggebend für den des Motorspannung. Da die Motorspannung nicht weiter steigen kann und der Motorstrom leicht absinkt, kann die Drehzahl analog zum sinkenden Strom durch die festgelegten Grenzen leicht steigen (Vgl. Gleichung \ref{eq:motorspannung}). Die Maximaldrehzahl ist damit auf \SI{12000}{RPM} begrenzt. 

%Zu Beginn des Steigfluges wären größere Steigwinkel effizienter, allerdings werden diese durch den maximalen Motorstrom begrenzt. Ohne diese würde der Bahnneigungswinkel beinahe linear absinken. Daraus kann geschlossen werden, dass ein Flug mit maximalem Motorstrom im unteren Höhenbereich am effizientesten ist. Der sägezahnartige Verlauf der Motorspannung hängt mit der gewählten Diskretisierung / Genauigkeit zusammen. Eine genauere Untersuchung dieser Punkte würde zu einem glatten Verlauf des Motorstroms bei \ensuremath{I_{max}} führen. Ebenso würde sich der Verlauf aller anderen Kurven glätten. Gleichzeitig zum konstanten Motorstrom wächst die Motorspannung linear an, bis sie ab \SI{8100}{m} das Niveau der Batteriespannung erreicht. Damit ist die PWM bei \SI{100}{\%}. Ab diesem Zeitpunkt kann nicht mehr mit maximalem Motorstrom geflogen werden, \textcolor{red}{wodurch folglich der Motorspannung und der Bahnneigungswinkel simultan abfallen}. Der Zusammenhang ergibt sich daher, dass mit dem Bahnanstellwinkel der Schub und die Geschwindigkeit steigen (Vgl. Gleichung \ref{eq:schub_flaechenflugzeug} und \ref{eq:schub_flaechenflugzeug}). Die im Anschluss aus dem Kennfeld interpolierte Drehzahl fließt direkt in den Motorstrom ein (Vgl. Gleichung \ref{eq:motorstrom}). Bedingt durch die mit dem Winkel und der Höhe (indirekt durch die Dichte) steigende Geschwindigkeit, können die Steigwinkel leistungsbedingt nicht mehr aufgebracht werden. Die maximale Motorspannung entspricht hier der maximalen Batteriespannung, die durch die Last von anfänglich \SI{15,6}{V} auf ca. \SI{14.7}{V} einbricht. Der Verlauf des Batteriestroms steht in direktem Zusammenhang mit dem Motorstrom und der Motorspannung. Dies wird aus Gleichung \ref{eq:batteriestrom} ersichtlich. Bei konstanten Motorstrom ist die \ensuremath{I_{Bat}} nur von \ensuremath{U_{Mot}} abhängig. Daher der gleiche Verlauf wie bei \ensuremath{U_{Mot}}. Danach ist \ensuremath{U_{Mot}} konstant und \ensuremath{I_{Bat}} hängt nur noch von \ensuremath{I_{Mot}} ab.




\subsection{Einflussfaktoren auf das Flächenflugzeug}

\subsubsection{Motor-Propeller Kombination}
Die Motor-Propeller-Kombination beeinflusst entscheidend das Leistungsverhalten von elektrisch, propellergetriebenen Fluggeräten. Mit dem in Tab. \ref{tab:flzg_parameter} aufgeführten Motor mit einem Gewicht von \SI{106}{g} und einem \ensuremath{K_V}-Wert von \SI{1390}{RPM/V} sind bereits sehr hohe Flughöhen erreichbar. Bei Verwendung des gleichen Propellers, einem 9x7 Propeller, und der Variation des \ensuremath{K_V}-Wertes, zeigen die Motoren mit einem größeren \ensuremath{K_V}-Wert ein besseres Flugverhalten. Die optimale Flugweise des Flächenflugzeuges ist für jede Art des Motors mit gleichem Gewicht identisch. Zuerst wird solange mit maximalem Motorstrom geflogen, bis die Schubhebelstellung \SI{100}{\%} erreicht. Danach sinkt der Bahnneigungswinkel während die Steiggeschwindigkeit steigt. Die maximale Höhe ist erreicht, wenn der noch fliegbare Steigwinkel Null erreicht und kein Steigflug mehr möglich ist. Der Steigwinkel leigt dabei in einem Bereich von \SI{15}{^\circ} und \SI{20}{^\circ}.
\textcolor{red}{Aus dem Motoren-Buch die Beziehung zwischen KM und KV Wert anführen}
An dieser Stelle ist auch die Motor-Propeller Kombination zu beachten. Der Motor mit einem \ensuremath{K_V}-Wert von 2850 erzielt mit dem 9x7 Propeller zwar signifikant schlechtere Flugeigenschaften, erreicht mit einem 6x4 Propeller noch größere Höhen. Zusammengenommen zeigt sich, dass mit geringer werdenden \ensuremath{K_V}-Wert, also einem langsamer, aber mit höherem Drehmoment drehender Motor, der optimale Durchmesser des Propellers in reziproker Weise steigt bei einem gleichen Verhältnis zwischen Durchmesser und Pitch. Dies kann relativ einfach mit den Angaben der Hersteller zu der besten Motor-Propeller-Kombination verglichen werden. Nach \cite{Wall.2015} erhöht sich der Wirkungsgrad eines Rotors mit größer werdenden Durchmesser. 
Mit den oben gemachten Aussagen zu einer Motor-Propeller-Kombination wird im Folgenden der Propeller an die Wahl der Motoren angepasst.
\textcolor{red}{hier Kurvenschar mit gleichem Motorgewicht, aber anderen KV, evtl. Prop anpassen}
\subsubsection{Anzahl der Motoren und Propeller}
Während die Leistung der Motoren mit gleichem Gewicht wenig Einfluss auf den optimalen Steigwinkel hat, ändert sich dies bedeutend mit der Anzahl der Motoren. Schon mit einer Steigerung der Motorenanzahl auf 2 verändert sich der optimale Steigwinkel zu \SI{90}{^\circ}. Die dazu zugehörige Steiggeschwindigkeit liegt hierbei beim Maximum der Steiggeschwindigkeitsiterationsweite. Dies ist solange der optimale Betriebspunkt bis der Steigwinkel von \SI{55}{^\circ} als optimaler ist. Danach ergibt sich das gleiche Verhalten wie es in \ref{subsec:erste_untersuchung} beschrieben ist. Das gleiche Flugverhalten ist bei einer Erhöhung der Anzahl auf 4 zu beobachten.


Wird die Anzahl der Propeller erhöht, so liegt der effizienteste Flug eines Flächenflugzeuges im vertikalen Steigflug mit \ensuremath{\gamma = 90^\circ}. Außerdem erhöht sich die mit Anzahl an Motoren auch der TOC
\subsubsection{Gleitzahl}
\subsubsection{Auslegungsgeschwindigkeit}
\subsubsection{Penaltyfaktor}





\subsubsection{Motor-Propeller-Kombination}
Die Motor-Propeller-Kombination beeinflusst entscheidend das Leistungsverhalten von elektrisch, propellergetriebenen Fluggeräten. Für jeden Motor werden vom Hersteller Propeller für einen bestimmten Anwendungsfall vorgegeben, mit dem optimale Leistungen erbracht werden können. Die Propellergröße hängt von der Leistung des Motors ab und von dem Auslegungsfall. Es zeigt sich, dass je stärker der Motor ist, desto größer ist der optimale Radius. Prinzipiell können auch kleinere Propeller verwendet werden. Aufgrund der Motorleistung ist der Pitch deshalb groß zu wählen, um den erforderlichen Schub zu liefern. Wird der Radius weiter vergrößert, so verringert sich der optimale Pitch. Außerdem sind Motoren mit niedrigen \ensuremath{K_V}-Werten bei gleichem Motorgewicht zu bevorzugen. Dies liegt in der Tatsache begründet, dass der \ensuremath{K_V}-Wert direkt den Motorstrom beeinflusst und ein hoher Wert diesen entsprechend zu Beginn des Fluges bereits stark erhöht. Zudem steht ein hoher \ensuremath{K_V} für eine hohe Maximaldrehzahl des Motors, aber für entsprechend weniger Drehmoment. Das gleiche gilt umgekehrt. 

%\begin{itemize}
%%	\item Motor-Propeller Kombination hat signifikanten Einfluss auf das Leistungsverhalten
%	\item jeder Motor hat einen Propeller, mit dem er optimalen Schub liefert
%	\item es zeigt sich, dass für einen schwächeren Motor der Durchmesser und der Pitch sinkt
%	\item Je stärker der Motor ist desto größer ist der optimale Radius. Prinzipiell können auch kleinere Propeller verwendet werden. Aufgrund der Motorleistung ist der Pitch auch groß zu wählen. 
%	\item Je größer der Durchmesser, desto kleiner darf der verwendete fixed Pitch gewählt werden. 	
%\end{itemize}

\subsubsection{Anzahl der Motoren}
Noch bessere Ergebnisse können mit 2 Motoren erreicht werden, die  einen niedrigen \ensuremath{K_V}-Wert besitzen, aber ein hohes Leistungsgewicht. Mit dieser Konstellation sind Flughöhen bis zu \SI{15000}{m} möglich. In diesem Fall ist nicht der Flug mit maximalen Motorstrom am effizientesten sonder ein Flug mit konstantem Steigwinkel von ca. \SI{60}{^\circ}. Dabei sinkt der Motorstrom in diesem Zustand leicht, da das Drehmoment mit der Höhe abnimmt. Ebenfalls wie oben ist beschrieben ist dieser Zustand so lange fliegbar bis der Motorstrom auf dem Niveau des Batteriestroms und damit \SI{100}{\%} der PWM erreicht ist. Ab diesem Punkt steigt der Bahnneigungswinkel wieder an, da für einen Höhenschritt die Fluggeschwindigkeit mit Winkel sinkt. alle anderen Größen verhalten sich analog zum oben beschriebenen Zustand


\subsubsection{Gleitzahl}
Mit einer Verringerung der Gleitzahl geht auch eine Verringerung der maximalen Höhe mit einher und vice versa. Hierbei verändert sich der Verlauf des Steigwinkels nicht ausschlaggebend. Eine entsprechend hohe Gleitzahl beudeuted gleichzeitig auch eine entsprechend hohe aerodynamische Güte (Vgl. \cite{Scheiderer.2008}). Dazu sinkt der Widerstand im Vergleich zum Auftrieb, sodass für ein Flächenflugzeug mit einer höheren Gleitzahl für den gleichen Auftrieb weniger Leistung zur Kompensation des Widerstandes aufgebracht werden muss. Als Konsequenz dessen steht mehr Leistung für das Steigen zur Verfügung.

%\begin{itemize}
%	\item mit einer Verringerung der Gleitzahl geht auch eine Verringerung der maximalen Höhe mit einher
%	\item eine Erhöhung erzeugt gleichzeitig eine Erhöhung des Top of Climb
%	\item dabei ändert sich der Verluaf des Steigwinkels nicht ausschlaggebend
%	\item genau wie zuvor ist ab einem gewissen Punkt kein Steigen mehr möglich, der Bahnneigungswinkel geht gegen Null
%	\item ein größerer Steigwinkel stellt auch eine größere aerodynamische Güte dar, direkt ersichtlich \cite{Scheiderer.2008}
%\end{itemize}

\subsubsection{Auslegungsgeschwindigkeit}
Die Auslegungsgeschwindigkeit hat einen bedeutenden Einfluss auf die erreichbare Höhe. Da für den Steigflug ein Flug mit konstanten Auftriebsbeiwert vorausgesetzt wird, erhöht sich aufgrund dessen die absolute Fluggeschwindigkeit mit der Höhe und größerem Bahnneigungswinkel (Vgl. Gleichung \ref{eq:geschw_flaechenflugzeug}). Ist die Auslegungsgeschwindigkeit gering, so wächst sie absolut gesehen mit der Höhe nicht so stark wie hohe Geschwindigkeiten. Eine geringer gewählte Auslegungsgeschwindigkeit im Horizontalflug bedeutet daher auch, dass länger mit maximalen Motorstrom geflogen werden kann, bevor die Motorspannung die Batteriespannung erreicht und somit das Absinken des Steigwinkels einleitet.


%\begin{itemize}
%	\item die Auslegungsgeschwindigkeit hat bedeutenden Einfluss auf die Erreichbare Höhe
%	\item das hängt damit zusammen, dass die Fluggeschwindigkeit für Winkel \ensuremath{\gamma >0} mit der Höhe und dem Winkel steigt
%	\item eine geringer gewählte Auslegungsgeschwindigkeit im Horizontalflug bedeutet, dass länger mit maximalen Motorstrom geflogen werden kann, bevor die Motorspannung die Batteriespannung erreicht und somit das Absinken des Steigwinkels einleitet
%	\item höhere Auslegungsgeschwindigkeiten haben einen entsprechend umgekehrten Einfluss und verringern die max erreichbar Höhe, da die Fluggeschwindikeit entsprechend immer höher wird, bis die Leistung des Flugzeugs nicht mehr ausreicht
%\end{itemize}

\subsubsection{Auslegungsgleitzahl}
Eine verringerte Gleitzahl ist nur im begrenzten Maße möglich. Mit kleiner werdender Auslegungsgleitzahl steigt 
der Widerstand im Vergleich zum Auftrieb, der im Horizontalflug konstant bleibt. Als Folge steigt auch der Nullwiderstand. Somit befindet sich das Flächenflugzeug aerodynamisch gesehen vermehrt in der Grauzone, sodass die Aussagekraft der Ergebnisse angezweifelt werden muss. 

\subsection{Ergebnisse des Vergleichs} 
\begin{itemize}
	\item Ergebnisse 
	\item ein Flächenflugzeug mit einem Motor erweist sich effizienter als
ein Quadrocopter in Bezug auf die max. erreichbare Höhe
	\item bei Vernachlässigung von zusätzlichen Widerständen und unter
Berücksichtigung des einfachen Modells ist dieser Vorteil gerade
wieder hinfällig sprich der zusätzliche Höhengewinn schwindet zu Null, wenn man die Widerstände berücksichtigt
	\item weiterhin erweist sich das Flächenflugzeug als bereits in den möglichen Maßen optimiert (Steiggeschwindigkeit ist in Bezug auf Auslegungszustand und bei Auslegungsgleitzahl optimal, Steigwinkel wird optimiert, Auslegungsgeschwindigkeit und optimale Konstellation von Motor und Propeller
	\item der Quadrocopter zeigt in diese Richtung noch Potenzial
	\item bei der Erhöhung der Motoren- und damit auch Propelleranzahl neigt das Flugzeug dazu in einem 90° Winkel zu steigen, dabei optimiert sich die Steigeschwindigkeit bisher automatisch
- eine Erhöhung der Gleitzahl erhöht die Anzahl der Ausreißer, quasi Abweichungen von dem 90° Zustand (zwischenzeitlich ist Flug mit 90° ernergieoptimaler, zwischenzeitlich der Gleitflug/ Steigflug mit geringerem Steigwinkel)
	\item letztendlich führt die bisherige Untersuchung wohl zu dem Design eines VTOL-Fliegers (Vermutung)
	\item wir haben weiterhin ein aerodynamisches Modell besprochen für einen VTOL-Flieger, Flugzustand bei entsprechendem Wind (Ausrichtung bspw. Messerflug) und Geschwindigkeitsmodell, entsprechend 4 mögliche Zustände (Steigwinkel anpassen, Steigwinkel und Fluggeschwindigkeit aus der Gleichungssystem lösen, etc.)
	\item ich werde nun den Multicopter weiter untersuchen (Anpassung der Steiggeschwindigkeit, Gewicht, Größe, etc.)
	\item als unterer Ast des Baumes ist eine Untersuchung der Verkleidung des Copters in Richtung VTOL-Flugzeug interessant,folglich auch weitere Untersuchung in diese Richtung

\end{itemize}



\section{Steiggeschwindigkeit}
\label{sec:steiggeschwindigkeit}
Eine weitere Optimierung des Multicopters bzw. des Quadrocopters kann durch eine Anpassung der Steiggeschwindigkeit geschehen. Die vormalig als konstant angenommene Steiggeschwindigkeit von \SI{10}{m/s} ist nicht in jedem Operationspunkt die optimale. Die Steiggeschwindigkeit wird wieder für jeden Höhenschritt variiert. Analog zur Variation des Steigwinkels beim Flächenflugzeug fällt die Auswahl der Geschwindigkeit auf den Wert, welcher die geringste Energiemenge benötigt für den Aufstieg. Bei der Untersuchung kristallisieren sich drei starke Einflussfaktoren heraus. Im Einzelnen sind das der Widerstandsbeiwert, die Anzahl der Batteriezellen und die Motorleistung

\subsection{Einfluss des Widerstands}
Der Widerstandsbeiwert hat einen entscheidenden Einfluss auf die maximale Steiggeschwindigkeit. Bei einem großen maximalen Motorstrom gilt, dass die Begrenzung der Geschwindigkeit durch den Widerstandsbeiwert erfolgt. Eine sehr hohe Geschwindigkeit verringert zum einen die Flugzeit für einen Höhenbereich, erhöht auf der anderen Seite jedoch den Widerstand und damit zusätzlich die benötigte Leistung. Je geringer der \ensuremath{C_W} gewählt wird, desto höher ist die optimale Steiggeschwindigkeit. Erhöht sich im Umkehrschluss der Luftwiderstand so sinkt die Steiggeschwindigkeit, da der Widerstand mit der Geschwindigkeit quadratisch (Vgl. Gleichung \ref{eq:widerstand}) ansteigt. Im Sinne einer großen maximalen Höhe ist daher eine aerodynamisch günstige Verkleidung des Multicopters anzustreben.  


\begin{itemize}
	\item der Widerstandsbeiwert hat einen entscheidenden Einfluss auf die maximale Steiggeschwindigkeit
	\item prinzipiell gilt bei einem großen maximalen Motorstrom, dass die Begrenzung der Geschwindigkeit durch den Widerstandsbeiwert erfolgt. Eine sehr hohe Geschwindigkeit verringert zum einen die Flugzeit für einen Höhenbereich, erhöht auf der anderen Seite jedoch den Widerstand und damit die benötigte Leistung. 
	\item je geringer der \ensuremath{C_W} gewählt wird, desto höher die Steiggeschwindigkeit, da der Widerstand bei großen Werten noch einen geringen Einfluss hat
	\item Eine Erhöhung von diesem bezweckt eine Erhöhung eine Verringerung der optimalen Fluggeschwindigkeit
	\item auch hier gilt wieder, dass der Flug mit \SI{100}{\%} am effizientesten ist
\end{itemize}

\subsection{Einfluss der Anzahl der Batteriezellen}
Ein weiterer begrenzender Parameter ist die PWM. Die Motorspannung an sich ksann nicht beeinflusst werden. Jedoch lässt sich Einfluss auf die Höhe der Motorspannung durch eine Erhöhung der in Reihe geschalteten Batteriezellen nehmen. Mit jeder zusätzlichen Zelle erhöht sich die Batteriespannung um \SI{3,7}{V}. Damit stellt die PWM nicht mehr die Grenze für die Steiggeschwindigkeit dar. Der effizienteste Flugzustand ist nun der bei maximalen, dauerhaften Motorstrom. Jedoch verringert eine höhere Batteriekapazität. Bei gleicher Energiemenge 
\begin{equation}
	Wh = Ah\cdot V
\end{equation}
führt eine Erhöhung der Spannung in dem Produkt aus Spannung und Kapazität unweigerlich zu einer Verringerung der Kapazität. Die schlägt sich wieder auf den Kostenfaktor aus, der erreichbaren Flughöhe. Diese Maßnahme ist also mit Bedacht zu wählen. Eine extreme Erhöhung der Zellenanzahl bewirkt außerdem wieder ein Flug mit maximalen Motorstrom.

\begin{itemize}
	\item Einfluss der Batteriezellenanzahl
	\item als begrenzender Parameter erweist sich die PWM
	\item dieser kann ausgewichen werden, wenn die Anzahl der Zellen erhöht wird, weil damit die Batteriespannung steigt. 
	\item damit begrenzt nicht mehr die PWM die Höhe, sondern es sind andere Parameter
	\item Wenn sie nicht mehr begrenzt, wird als erstes die maximale Motorspannung erreicht und konstant mir ihr geflogen
	\item dies ist nicht so effizient, wie mit maximaler Leistung, entsprechend verringert sich die max. erreichbare Höhe
	\item Der Batteriezellenanzahlerhöhung sind Grenzen gesetzt
	\item Durch eine Erhöhung der Batteriespannung verringert sich um Umkehrschluss die Kapazität der Batterie, da gilt \ensuremath{Wh = Ah\cdot U}
	\item Eine extreme Erhöhung der Zellenanzahl bewirkt außerdem wieder ein Flug mit maximalen Motorstrom
\end{itemize}

\subsection{Einfluss des maximalen Motorstroms}
Die Ergebnisse zeigen, dass ein geringer maximaler Motorstrom die ebenfalls die Steiggeschwindigkeit begrenzt. Dieser begrenzt die dem Motor entnommene Leistung. Folglich ist ein Motor für einen solchen Steigflug zu wählen, der einen entsprechend hohen Dauerstrom vertragen kann, wie der Motor aus Kapitel \ref{chap:nachbildung}.

Die Motorleistung
\begin{itemize}
	\item Einfluss des maximalen Motorstroms
	\item ist dieser sehr gering gewählt so stellt er den begrenzenden Parameter
	\item dieser begrenzt die dem Motor entnommene Leistung 
	\item entsprechend ist diese groß zu wählen wie beim Cobra-Motor
\end{itemize}

\section{Verstellpropeller}
\label{sec:verstellprop}

\begin{itemize}
	\item Aus der Datenbank von APC werden alle Propeller mit dem vorgegebenen Durchmesser und verschiedenen Pitches entnommen
	\item Vernachlässigt werden dabei gesonderte Propeller mit z.B. besonderem Pitch, Verwindung, E am Ende oder mit mehr als 2 Blättern (3 oder 4)
	\item \textcolor{red}{hier interessant, ob Funktion für Regression des Wirkungsgrades über dem Schubbeiwert???}
	\item interessant ist hier Regression --> Frage nur über was, Wirkungsgrad in Abhängigkeit von \ensuremath{C_T} oder \ensuremath{\lambda}
	\item Auswahl nach 
	\begin{equation}
		\eta_{Prop} = \frac{T\cdot\tilde{V}}{P} = \frac{T\cdot (\mu_z + v_i)}{M\cdot \omega}
	\end{equation}
	\item Cite Ogur
\end{itemize}

\section{Steigwinkel}
\label{sec:steigwinkel}

\section{Größe des Fluggerätes}
\label{sec:groesse}

\section{VTOL im Vergleich zum Gleitflug}
\label{sec:vtol_vs_gleitflug}
\begin{itemize}
	\item hier ist der Vergleich vor allem in Bezug auf die Motorisierung interessatn
	\item höhere Motorisierung und zus. Gewicht <--> schächere Motorisierung und geringeres Gewicht im Gleitflug
\end{itemize}