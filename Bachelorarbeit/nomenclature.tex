% ******************************************************************************
% Beispiele zur Nutzung des glossaries Paketes.
%
% Ein Beispiel, um ein eigenes Symbolverzeichnis zu erstellen:
%\newglossary[slg]{symbolslist}{syi}{syg}{Symbolverzeichnis}
%
% Diese Befehle sortieren die Eintraege in den einzelnen Listen, wobei zuvor
% pdflatex ausgefuehrt werden muss. Für die Erstellung und Sortierung der Listen
% sollte das zur Verfuegung stehende Makefile verwendet werden.
% pdflatex
% makeindex -s datei.ist -t datei.llg -o datei.lyi datei.lyg
% makeindex -s datei.ist -t datei.glg -o datei.gyi datei.gyg
% makeindex -s datei.ist -t datei.ilg -o datei.iyi datei.iyg
% makeindex -s datei.ist -t datei.slg -o datei.syi datei.syg
% makeindex -s datei.ist -t datei.mlg -o datei.myi datei.myg
% makeindex -s datei.ist -t datei.alg -o datei.acr datei.acn
% makeindex -s datei.ist -t datei.glg -o datei.gls datei.glo
% pdflatex
% pdflatex
%
% Für das Programm TeXmaker ist folgender Befehl in den Optionen zu definieren:
% pdflatex %.tex | makeindex -s %.ist -t %.llg -o %.lyi %.lyg | makeindex -s %.ist -t %.glg -o %.gyi %.gyg | makeindex -s %.ist -t %.ilg -o %.iyi %.iyg | makeindex -s %.ist -t %.slg -o %.syi %.syg | makeindex -s %.ist -t %.mlg -o %.myi %.myg | makeindex -s %.ist -t %.alg -o %.acr %.acn | makeindex -s %.ist -t %.glg -o %.gls %.glo | pdflatex %.tex | pdflatex %.tex
%
% Die coolen VIM Nutzer können die Makefile oder latexmk in die vim-latex-suite
% einbinden.
% ******************************************************************************
%
% Erstellen von verschiedenen Symbolverzeichnissen fuer diese Arbeit
%
% Lateinische Zeichen
\newglossary[llg]{latin}{lyi}{lyg}{Lateinische Bezeichnungen}
% Griechische Zeichen
\newglossary[glg]{greek}{gyi}{gyg}{Griechische Bezeichnungen}
% Indizes
\newglossary[ilg]{indices}{iyi}{iyg}{Indizes}
% ******************************************************************************
%
% Glossar Einstellungen
%
% Den Punkt am Ende jeder Beschreibung deaktivieren
\renewcommand*{\glspostdescription}{}
%
% Ein dreispaltiges symbolverzeichnis definieren mit den Eintraegen:
% Notation | Einheit | Beschreibung
\newglossarystyle{symbol}{
    \glossarystyle{long3colheader}
    \renewenvironment{theglossary}{%
        \begin{longtable}{lp{2.5cm}p{1.5\glsdescwidth}}
    }{%
        \end{longtable}
    }
    \renewcommand*{\glossaryheader}{%
        \textbf{Notation} & \textbf{Einheit} & \textbf{Beschreibung}\\
    }%
    \renewcommand*{\glossaryentryfield}[5]{%
        \glsentryitem{##1}\glstarget{##1}{##2} & ##4 & ##3  \\
    }%
}
%
% Ein zweispaltiges symbolverzeichnis definieren mit den Eintraegen:
% Notation | Beschreibung
\newglossarystyle{indice_symbol}{
    \glossarystyle{longheader}
    \renewenvironment{theglossary}{%
        \begin{longtable}{p{2.5cm}p{1.7\glsdescwidth}}
    }{%
        \end{longtable}
    }
    \renewcommand*{\glossaryheader}{%
        \textbf{Notation} & \textbf{Beschreibung}\\
    }%
    \renewcommand*{\glossaryentryfield}[5]{%
        \glsentryitem{##1}\glstarget{##1}{##2} & ##3  \\
    }%
}
%
% Glossar-Befehle anschalten
\makeglossaries
% ******************************************************************************
% Befehle für Symbole
% ******************************************************************************
%
% Lateinische Bezeichnungen
% ******************************************************************************
\newglossaryentry{symb:x}{
    name=$\vv{x}$,
    description={Zustandsvektor eines Zustandsraummodells, der Form
    \mbox{
	    $\dot{\vec{x}} = \vec{f}(\vec{x}) + \vec{g}(\vec{x}) u
	    \text{,}\quad
    	y = h(\vec{x})$
    }
    },
    symbol={-},
    sort=symbolx,
    type=latin
}
\newglossaryentry{symb:y}{
    name=$y$,
    description={Ausgangsgröße},
    symbol={-},
    sort=symboly,
    type=latin
}
\newglossaryentry{symb:APQ_lyp}{
	name={$\MM{A}, \MM{P}, \MM{Q}$},
	description={Positiv definite Matrizen der linearen \textsc{Lyapunov}
	Matrix-Gleichung
	},
    symbol={-},
    sort=symbolAPQ,
    type=latin
}
% ******************************************************************************
%
% Griechische Bezeichnungen
% ******************************************************************************
\newglossaryentry{symb:xi}{
    name=$\vv{\xi}$,
    description={Teilzustandsvektor der Zustände des transformierten Systems,
	    auch als externe Dynamik bezeichnet
    },
    symbol={-},
    sort=symbolxi,
    type=greek
}
\newglossaryentry{symb:nu_ad}{
	name=$\vv{\nu}_{ad}$,
    description={Adaptiver Anteil der Pseudosteuergröße},
    symbol={-},
    sort=symbolnu_ad,
    type=greek
}
\newglossaryentry{symb:sigmaSVD}{
	name={$\overline{\sigma}, \underline{\sigma}$},
    description={maximaler und minimaler Singulärwert},
    symbol={-},
    sort=symbolsigmamaxmin,
    type=greek
}
\newglossaryentry{symb:betagamma}{
	name={$\beta, \gamma$},
    description={Funktionen bestimmter monotoner Klassen},
    symbol={-},
    sort=symbolbetagamma,
    type=greek
}
% ******************************************************************************
%
% Indizes
% ******************************************************************************
\newglossaryentry{symb:inftyindex}{
    name={$\infty$},
    description={Größen der ungestörten Anströmung},
    sort=symbol.infty,
    type=indices
}
\newglossaryentry{symb:sub_latin}{
    name={$i, j, k, l$},
    description={Bezeichnen mit dem Index eine Komponente eines Tensors, während
        bei doppelten Index eine Summation der Komponenten nach der
        \textsc{Einstein}schen Summenkonvention erfolgt},
    sort=symbol_ijkl,
    type=indices
}
\newglossaryentry{symb:sub_greek}{
    name={$\alpha , \beta , \gamma$},
    description={Bezeichnen mit dem Index eine Komponente eines Tensors, auch
        wenn ein doppelter Index vorliegt},
    sort=symbol_alphabetagammadelta,
    type=indices
}
\newglossaryentry{symb:invariant1}{
    name={$\textrm{I}$},
    description={Bezeichnen mit dem Index die erste Invariante des jeweiligen
        Tensors
    },
    sort=symbol.1,
    type=indices
}
\newglossaryentry{symb:invariant2}{
    name={$\textrm{I}\textrm{I}$},
    description={Bezeichnen mit dem Index die zweite Invariante des jeweiligen
        Tensors
    },
    sort=symbol.2,
    type=indices
}
\newglossaryentry{symb:invariant3}{
    name={$\textrm{I}\textrm{I}\textrm{I}$},
    description={Bezeichnen mit dem Index die dritte Invariante des jeweiligen
        Tensors
    },
    sort=symbol.3,
    type=indices
}
% ******************************************************************************
%
% Abkuerzungen
% ******************************************************************************
%\newacronym[description=So wie hier steht es im Verzeichnis]{abc}{abc}{So wie hier steht es im Text}
%
\newacronym{TU-Braunschweig}{TU-Braunschweig}{
	Technischen Universit\"{a}t Braunschweig
}
\newacronym{IFF}{IFF}{Institut für Flugführung}
\newacronym{NASA}{NASA}{National Aeronautics and Space Administration}
% ******************************************************************************
%
% Abkurzunge mit Glossareintrag
% ******************************************************************************
\newacronym{SISO}{SISO}{Single Input Single Output}
\newacronym{FAA}{FAA}{Federal Aviation Administration}
\newacronym{VIM}{VIM}{\texttt{V}i \texttt{IM}proved}
% ******************************************************************************
%
% Glossareintraege
% ******************************************************************************
\newglossaryentry{glos:SISO}{
    name=Single Input Single Output,
    description={Eingrößensystem
    }
}
\newglossaryentry{glos:Deployment}{
    name=Software Deployment,
    description={Das Software Deployment umfasst die gesamten
	    Entwicklungsaktivitäten die den Einsatz der Software ermöglichen.
    }
}
\newglossaryentry{glos:VIM}{
    name={\texttt{V}i \texttt{IM}proved},
    description={Einer der essenziell wichtigsten Texteditoren des Universums.
 	\texttt{V}i \texttt{IM}prove dist eine Weiterentwicklung des Texteditors
	vi und funktioniert wie der vi-Editor im Textmodus auf jedem Terminal!
    }
}
% ******************************************************************************
