\chapter{Projektmanagement}
\label{projektmanagement}

%%%%%%%%%%%%%%%%%%%%%%%%%%%%%%%%%%%%%%%%%%%WBS%%%%%%%%%%%%%%%%%%%%%%%%%%%%%%%%%%%%%%%%%%%%

\begin{landscape}
\section{Projektstrukturplan}
% 	\captionof{Projektstrukturplan}
\label{WBS}
        \vspace{0.5cm}
        \begin{tikzpicture}[node distance=0.7cm]
            \tikzstyle{abstract}=[rectangle, draw=black, rounded corners, fill=blue!50, drop shadow,
        text centered, anchor=north, text=white, text width=3.5cm]
    \tikzstyle{topabstract}=[rectangle, draw=black, rounded corners, fill=blue!20, drop shadow,
        text centered, anchor=north, text width=0.8\linewidth]
    \tikzstyle{subabstract}=[rectangle, draw=black, rounded corners, fill=blue!20, drop shadow,
        text centered, anchor=north, text width=3.5cm]
    \tikzstyle{myarrow}=[->, >=open triangle 90, very thick]
    \tikzstyle{line}=[-, very thick]
            \node (title) [topabstract]
            {\textbf{\Large \otitle }};

            \node (AuxNode03) [text width=3.5cm, below=of title] {};
            \node (AuxNode02) [text width=3.5cm, left=of AuxNode03] {};
            \node (AuxNode01) [text width=3.5cm, left=of AuxNode02] {};
            \node (AuxNode04) [text width=3.5cm, right=of AuxNode03] {};
            \node (AuxNode05) [text width=3.5cm, right=of AuxNode04] {};
            \node (AuxNode06) [text width=3.5cm, right=of AuxNode05] {};

            \node (ap3000) [abstract, rectangle split, rectangle split parts=2, below=of AuxNode03]
                {
                    \textbf{AP 3000}
                    \nodepart{second}Überprüfung und Valdierung des Programms
                };
            \node (ap2000) [abstract, rectangle split, rectangle split parts=2, below=of AuxNode02]
                {
                    \textbf{AP 2000}
                    \nodepart{second}Programmaufbau
                };
            \node (ap1000) [abstract, rectangle split, rectangle split parts=2, below=of AuxNode01]
                {
                    \textbf{AP 1000}
                    \nodepart{second}Literaturrecherche
                };
            \node (ap4000) [abstract, rectangle split, rectangle split parts=2, below=of AuxNode04]
                {
                    \textbf{AP 4000}
                    \nodepart{second}Parameteruntersuchung
                };
            \node (ap5000) [abstract, rectangle split, rectangle split parts=2, below=of AuxNode05]
                {
                    \textbf{AP 5000}
                    \nodepart{second}Auswertung
                };
            \node (ap1100) [subabstract, rectangle split, rectangle split parts=2, below=of ap1000]
                {
                    \textbf{AP 1100}
                    \nodepart{second}Flugmechanik und Aerodynamik von Multicoptern und Flächenflugzeugen
                };
            \node (ap1200) [subabstract, rectangle split, rectangle split parts=2, below=of ap1100]
                {
                    \textbf{AP 1200}
                    \nodepart{second}Zusammenhang einer elektromechanischen Antriebseinheit
                };
            \node (ap2100) [subabstract, rectangle split, rectangle split parts=2, below=of ap2000]
                {
                    \textbf{AP 2100}
                    \nodepart{second}Erstellung eines Modells in Matlab
                };
            \node (ap2200) [subabstract, rectangle split, rectangle split parts=2, below=of ap2100]
                {
                    \textbf{AP 2200}
                    \nodepart{second}Programmablauf
                };
            \node (ap3100) [subabstract, rectangle split, rectangle split parts=2, below=of ap3000]
               {
                   \textbf{AP 3100}
                   \nodepart{second}Nachbildung eines Quadrocopterfluges im Programm
               };
               \node (ap3200) [subabstract, rectangle split, rectangle split parts=2, below=of ap3100]
               {
                   \textbf{AP 3200}
                   \nodepart{second}Diskussion der Flugleistungen
               };
            \node (ap4100) [subabstract, rectangle split, rectangle split parts=2, below=of ap4000]
               {
                   \textbf{AP 4100}
                   \nodepart{second}Festlegung der zu untersuchenden Parameter
               };
            \node (ap4200) [subabstract, rectangle split, rectangle split parts=2, below=of ap4100]
               {
                   \textbf{AP 4200}
                   \nodepart{second}Programmerweiterung
                   };
            \node (ap4300) [subabstract, rectangle split, rectangle split parts=2, below=of ap4200]
                {
                    \textbf{AP 4300}
                    \nodepart{second}Einflussuntersuchung der Parameter
                };
            \node (ap5100) [subabstract, rectangle split, rectangle split parts=2, below=of ap5000]
                {
                    \textbf{AP 5100}
                    \nodepart{second}Diskussion der Ergebnisse
                };
%			 \node (ap5200) [subabstract, rectangle split, rectangle split parts=2, below=of ap5100]
%                {
%                    \textbf{AP 5200}
%                    \nodepart{second}Ausblick
%                };
            \node (AuxNode06) [text width=3.5cm, below=of ap5100] {};
            \node (ap6000) [abstract, rectangle split, rectangle split parts=2, below=of AuxNode06]
                {
                    \textbf{AP 6000}
                    \nodepart{second}Dokumentation
                };

            \draw[myarrow]  (ap5000.north) -- ++(0,0.8) -| (title.south);
            \draw[line] (ap5000.north) -- ++(0,0.8) -| (ap4000.north);
            \draw[line] (ap5000.north) -- ++(0,0.8) -| (ap3000.north);
            \draw[line] (ap5000.north) -- ++(0,0.8) -| (ap2000.north);
            \draw[line] (ap5000.north) -- ++(0,0.8) -| (ap1000.north);
            \draw[line] (ap5000.north) -- ++(0,0.8) -| (ap1000.north);
            \draw[line] (ap1000.west)  -- ++(-0.19,0);
            \draw[line] (ap1100.west)  -- ++(-0.2,0);
            \draw[line] (ap1200.west)  -- ++(-0.2,0) -- ([yshift=0.0cm, xshift=-0.19cm] ap1000.west);
            \draw[line] (ap2000.west)  -- ++(-0.19,0);
            \draw[line] (ap2100.west)  -- ++(-0.2,0);
            \draw[line] (ap2200.west)  -- ++(-0.2,0) -- ([yshift=0.0cm, xshift=-0.19cm]ap2000.west);
			\draw[line] (ap3000.west) -- ++(-0.19,0);
            \draw[line] (ap3100.west) -- ++(-0.2,0);
            \draw[line] (ap3200.west) -- ++(-0.2,0) -- ([yshift=0.0cm, xshift=-0.19cm]ap3000.west);
            \draw[line] (ap4000.west) -- ++(-0.19,0);
            \draw[line] (ap4100.west) -- ++(-0.2,0);
            \draw[line] (ap4200.west) -- ++(-0.2,0);
            \draw[line] (ap4300.west) -- ++(-0.2,0) -- ([yshift=0.0cm, xshift=-0.19cm] ap4000.west);
            \draw[line] (ap5000.west) -- ++(-0.19,0);
            \draw[line] (ap5100.west) -- ++(-0.2,0);

            \draw[line] (ap6000.east) -- ++(+0.2,0) |- ([yshift=0.8cm, xshift=0.00cm] ap5000.north);
        \end{tikzpicture}
%    \end{center}
%\end{sidewaystable}
\end{landscape}

%%%%%%%%%%%%%%%%%%%%%%%%%%%%%%%%%%%% Zeitplan %%%%%%%%%%%%%%%%%%%%%%%%%%%%%%%%%%%%%%%%%%%%%%

\begin{landscape}
\section{Zeitplan}
\label{sec:zeitplan}
\noindent\resizebox{\linewidth}{!}{	% Einfügen, falls zu groß
\begin{ganttchart}[hgrid,
                   time slot format = isodate,
                   x unit=0.2cm,	% Zum komprimieren des Charts in x-Richtung
                   y unit chart=1.05cm,
                   %compress calendar,	% Komprimiert den Chart in der Breite
                   calendar week text = {\currentweek},
                   chart element start border = right,
                   bar/.append style={fill=blue!40, rounded corners=2pt},
                   bar incomplete/.append style={fill=blue!10},
                   bar label node/.append style={align=left, text width=7cm},
                   group label node/.append style={align=left, text width=8cm},
                   milestone label node/.append style={align=left, text width=8cm},
                   bar progress label node/.style={right=2mm},
                   progress label text = {\pgfmathprintnumber[precision=0, verbatim]{#1}\%},
                  ]{2018-11-15}{2019-03-01}
  \gantttitlecalendar{year, month=shortname, week}\\
  %\gantttitle{2013}{59}\\
  \ganttgroup{AP 1000: Literaturrecherche}{2018-11-27}{2018-12-04}\\
  \ganttbar  {AP 1100: Flugmechanik und Aerodynamik von Multicoptern und Flächenflugzeugen}{2018-11-27}{2018-11-30}\\
  \ganttlinkedbar [link type=rdldr, link mid = 0.6, link bulge = 3]{AP 1200: Zusammenhang einer elektromechanischen Antriebseinheit}{2018-12-01}{2018-12-04}\\
  %\ganttbar[progress=100]{AP 1300: TEXT}{2013-01-01}{2013-01-30}\\	% Beispiel für Fortschrittsbalken!

  \ganttgroup{AP 2000: Programmaufbau}{2018-12-05}{2018-12-20}\\
  \ganttbar  {AP 2100: Erstellung eines Modells in Matlab}{2018-12-05}{2018-12-17}\\
  \ganttlinkedbar [link type=rdldr, link mid = 0.6, link bulge = 3] {AP 2200: Struktogramm}{2018-12-18}{2018-12-20}\\

  \ganttgroup{AP 3000: Überprüfung und Valdierung des Programms}{2018-12-21}{2018-12-28}\\
  \ganttbar  {AP 3100: Modellierung eines Quadrocopterfluges im Programm}{2018-12-21}{2018-12-22}\\
  \ganttlinkedbar [link type=rdldr, link mid = 0.6, link bulge = 3] {AP 3200: Diskussion der Flugleistungen}{2018-12-23}{2018-12-28}\\

  \ganttgroup{AP 4000: Parameteruntersuchung}{2019-01-02}{2019-02-07}\\
  \ganttbar{AP 4100: Festlegung der zu untersuchenden Parameter}{2019-01-02}{2019-01-09}\\
  \ganttlinkedbar [link type=rdldr, link mid = 0.6, link bulge = 3] {AP 4200: Programmerweiterung}{2019-01-10}{2019-02-07}\\
  \ganttlinkedbar [link type=rdldr, link mid = 0.6, link bulge = 3] {AP 4300: Einflussuntersuchung der Parameter}{2019-01-10}{2019-02-07}\\

  \ganttgroup{AP 5000: Auswertung}{2019-02-08}{2019-02-22}\\
  \ganttbar  {AP 5100: Diskussion der Ergebnisse}{2019-02-08}{2019-02-22}\\

  \ganttgroup{AP 6000: Dokumentation}{2018-12-04}{2019-02-22}\\
  %\ganttmilestone{Meilenstein}{2013-02-20}\\

 \end{ganttchart}
}
\end{landscape}


%%%%%%%%%%%%%%%%%%%%%%%%%%%%%%%%%%% WPD %%%%%%%%%%%%%%%%%%%%%%%%%%%%%%%%%%%%%%%%%%%%%%%%%%

% WPD 1100

\clearpage
\begin{wpd}{AP 1100}{Flugmechanik und Aerodynamik von Multicoptern und Flächenflugzeugen}{1.0}{Lucas Schreer}{26.11.2018}{27.11.2018}{31.11.2018}{5 Tage}{Lucas Schreer}
    {
    \textbf{Ziele:}
    \begin{itemize}
        \item grundlegende Berechnung der Flugleistungen eines Multicopters
        \item vereinfachte Berechnung der Flugleistungen eines Flächenflugzeugs
        \item Kenntnis über flugmechanische Zusammenhänge
    \end{itemize}
    \textbf{Input:}
    \begin{itemize}
        \item Literaturrecherche bezüglich der Flugmechanik und Aerodynamik von Hubschraubern und Flächenflugzeugen
    \end{itemize}
    \textbf{Schnittstellen zu anderen APs:}
    \begin{itemize}
        \item AP 2100
        \item AP 5200
    \end{itemize}
    \textbf{Aufgaben:}
    \begin{itemize}
        \item Literaturrecherche
        \item Einlesen in die Thematik der Aerodynamik von Hubschraubern sowie Flächenflugzeugen
    \end{itemize}
    \textbf{Ergebnisse:}
    \begin{itemize}
        \item Kenntnis über grundsätzliche flugmechanische und aerodynamische Zusammenhänge von Multicoptern bzw. Flächenflugzeugen
        \item Wissen über die Genauigkeit der getroffenen Annahmen sowie die Grenzen der Genauigkeit
    \end{itemize}
    }
\end{wpd}

% WPD 1200

\clearpage
\begin{wpd}{AP 1200}{Zusammenhang einer elektromechanischen Antriebseinheit}{1.0}{Lucas Schreer}{26.11.2018}{01.12.2018}{04.12.2018}{4 Tage}{Lucas Schreer}
    {
    \textbf{Ziele:}
    \begin{itemize}
        \item Kenntnis über elektromechanische Antriebseinheiten
        \item Wissen über die gegenseitige Beeinflussung der Antriebseinheiten
    \end{itemize}
    \textbf{Input:}
    \begin{itemize}
        \item Literaturrecherche bezüglich Brushlessmotoren, Reglern, Batterien , etc.
    \end{itemize}
    \textbf{Schnittstellen zu anderen APs:}
    \begin{itemize}
        \item AP 2100, AP 4000
    \end{itemize}
    \textbf{Aufgaben:}
    \begin{itemize}
        \item Auseinandersetzung mit der Thematik 
        \item Kenntnis über die Grundlagen eines elektromechanischen Antriebs        
    \end{itemize}
    \textbf{Ergebnisse:}
    \begin{itemize}
        \item Kenntnis über den Zusammenhang und die Berechnung einzelner Komponenten der elektrischen Antriebseinheit
        \item Wissen über die Grenzen der elektromechanischen Einheiten
    \end{itemize}
    }
\end{wpd}

%WPD 2100

\clearpage
\begin{wpd}{AP 2100}{Erstellung eines Modells in Matlab}{1.0}{Lucas Schreer}{27.11.2018}{05.12.2018}{17.12.2018}{2 Wochen}{Lucas Schreer}
    {
    \textbf{Ziele:}
    \begin{itemize}
        \item Implementierung der Flugmechanik und Aerodynamik von Multicoptern und Flächenflugzeugen in Matlab
        \item Implementierung des elektromechanischen Antriebsstrangs
    \end{itemize}
    \textbf{Input:}
    \begin{itemize}
        \item Ergebnisse aus AP 1100 und AP 1200
    \end{itemize}
    \textbf{Schnittstellen zu anderen APs:}
    \begin{itemize}
        \item AP 4300
    \end{itemize}
    \textbf{Aufgaben:}
    \begin{itemize}
    	\item Implementierung der Zusammenhänge zwischen Aerodynamik, Flugmechanik und der elektrischen Antriebseinheit
    	\item Anfertigen eines organisierten Programmablaufs von der Aerodynamik zur Batterieentladung
    	\item Darstellung der Ergebnisse in geeigneten Diagrammen
    \end{itemize}
    \textbf{Ergebnisse:}
    \begin{itemize}
        \item Ein geeignetes Programm für fortlaufende Untersuchungen und anschließende Programmerweiterung
        \item Fertiges Matlab Programm zur Durchführung einer ersten Simulationen von elektrisch angetriebenen Flugsystemen mit einer Bandbreite von Parametern, sowie deren Auswertung
    \end{itemize}
    }
\end{wpd}

% WPD 2200

\clearpage
\begin{wpd}{AP 2200}{Struktogramm}{1.0}{Lucas Schreer}{26.11.2018}{18.12.2018}{20.12.2018}{3 Tage}{Lucas Schreer}
    {
    \textbf{Ziele:}
    \begin{itemize}
        \item Erstellen eines Struktogramms für das Programm zur Leistungsberechnung
    \end{itemize}
    \textbf{Input:}
    \begin{itemize}
        \item Ergebnisse aus AP 2100
    \end{itemize}
    \textbf{Schnittstellen zu anderen APs:}
    \begin{itemize}
        \item AP 6000
    \end{itemize}
    \textbf{Aufgaben:}
    \begin{itemize}
        \item Erstellen eines Struktogramms für die einzelnen Programmabläufe
    \end{itemize}
    \textbf{Ergebnisse:}
    \begin{itemize}
        \item Strukturiertes Ablaufdiagramm, welches die entsprechenden Abläufe ohne Quelltext darstellt
    \end{itemize}
    }
\end{wpd}

% WPD 3100

\clearpage
\begin{wpd}{AP 3100}{Nachbildung eines Quadrocopterfluges im Programm}{1.0}{Lucas Schreer}{26.11.2018}{21.12.2018}{22.12.2018}{2 Tage}{Lucas Schreer}
    {
    \textbf{Ziele:}
    \begin{itemize}
        \item Überprüfung der Validität des Quadrocopterfluges in Russland
        \item Validierung des aufgestellten Modells
    \end{itemize}
    \textbf{Input:}
    \begin{itemize}
        \item Ergebnisse aus AP 2100
    \end{itemize}
%    \textbf{Schnittstellen zu anderen APs:}
%    \begin{itemize}
%        \item AP 3200
%    \end{itemize}
    \textbf{Aufgaben:}
    \begin{itemize}
        \item Internetrecherche aller benötigten Parameter zur Nachbildung des Fluges im Programm
        \item Darstellung der nachgebildeten Flugleistungen in Diagrammen
    \end{itemize}
    \textbf{Ergebnisse:}
    \begin{itemize}
        \item Nachgebildeter Flug im Programm
        \item 
    \end{itemize}
    }
\end{wpd}

% WPD 3200

\clearpage
\begin{wpd}{AP 3200}{Diskussion der Flugleistungen}{1.0}{Lucas Schreer}{26.11.2018}{23.12.2018}{28.12.2018}{3 Tage}{Lucas Schreer}
    {
    \textbf{Ziele:}
    \begin{itemize}
        \item Überprüfung der angegebenen Flugleistungen mit dem Programm
        \item Validierung des aufgestellten Modells
    \end{itemize}
    \textbf{Input:}
    \begin{itemize}
        \item Ergebnisse aus AP 3100
    \end{itemize}
    \textbf{Schnittstellen zu anderen APs:}
    \begin{itemize}
        \item AP 3100
    \end{itemize}
    \textbf{Aufgaben:}
    \begin{itemize}
        \item Abgleichen der Flugleistungen des Programms mit den im Video gezeigten
        \item Logische Prüfung der Ergebnisse in Bezug auf die Umsetzung
        \item Nachvollziehen und Klären der Plausibilität der im Video gezeigten Flugleistungen
    \end{itemize}
    \textbf{Ergebnisse:}
    \begin{itemize}
        \item Aussagen zur Validität des aufgestellten Modells
        \item Kenntnis über die 
    \end{itemize}
    }
\end{wpd}

% WPD 4100

\clearpage
\begin{wpd}{AP 4100}{Festlegung der zu untersuchenden Parameter}{1.0}{Lucas Schreer}{26.11.2018}{02.01.2019}{09.01.2019}{1 Woche}{Lucas Schreer}
    {
    \textbf{Ziele:}
    \begin{itemize}
        \item Liste mit allen zu untersuchenden und variierenden Parametern
        \item Wissen um die Implementierung im Modell
    \end{itemize}
    \textbf{Input:}
    \begin{itemize}
        \item Ergebnisse aus AP100, AP 2000 und AP 3000
    \end{itemize}
    \textbf{Schnittstellen zu anderen APs:}
    \begin{itemize}
        \item AP 4200 und AP 4300
    \end{itemize}
    \textbf{Aufgaben:}
    \begin{itemize}
        \item Herausfiltern relevanter Parameter
        \item Suchen nach Möglichkeiten zur Variation der Parameter
        \item Vorabeinschätzung der Relevanz für die Flugleistungen
        \item Klärung eventueller Interferenzen
    \end{itemize}
    \textbf{Ergebnisse:}
    \begin{itemize}
        \item Anzahl an zu untersuchenden Parameter und sinnvolle Variation dieser
        \item mögliche Zusammenhänge einzelner Parameter
        \item grobe Programmablaufsequenzen zur Untersuchung der Parameter 
    \end{itemize}
    }
\end{wpd}

% WPD 4200

\clearpage
\begin{wpd}{AP 4200}{Programmerweiterung}{1.0}{Lucas Schreer}{26.11.2018}{10.01.2019}{07.02.2019}{4 Wochen}{Lucas Schreer}
    {
    \textbf{Ziele:}
    \begin{itemize}
        \item Erweiterung und Anpassung des Programms um neue Aspekte der Parameteruntersuchung
    \end{itemize}
    \textbf{Input:}
    \begin{itemize}
        \item Ergebnisse aus AP 2000 und AP 4100
    \end{itemize}
    \textbf{Schnittstellen zu anderen APs:}
    \begin{itemize}
        \item AP 2100
    \end{itemize}
    \textbf{Aufgaben:}
    \begin{itemize}
        \item Einbau weiterer Programmstrukturen, die die Untersuchung der in AP 4100 aufgestellten Parameter ermöglichen
        \item Erweiterung des Programms um Strukturen zur Visualisierung der Ergebnisse
    \end{itemize}
    \textbf{Ergebnisse:}
    \begin{itemize}
        \item Erweitertes und an die Untersuchung angepasstes Programm 
        \item Funktionen und Iterationen, die eine Parameteruntersuchung ermöglichen  
    \end{itemize}
    }
\end{wpd}

% WPD 4300 

\clearpage
\begin{wpd}{AP 4300}{Einflussuntersuchung der Parameter}{1.0}{Lucas Schreer}{26.11.2018}{10.01.2019}{07.02.2019}{4 Wochen}{Lucas Schreer}
    {
    \textbf{Ziele:}
    \begin{itemize}
        \item Untersuchung des Einflusses der in AP 4100 festgelegten Parameter
    \end{itemize}
    \textbf{Input:}
    \begin{itemize}
        \item Ergebnisse aus AP 4100 und AP 4200
    \end{itemize}
    \textbf{Schnittstellen zu anderen APs:}
    \begin{itemize}
        \item AP 2000
    \end{itemize}
    \textbf{Aufgaben:}
    \begin{itemize}
        \item Untersuchung des Parametereinflusses auf die Flugleistungen des Flugsystems
        \item Darstellung dieses Einflusses in dafür geeigneten Diagrammen, Graphen, Bildern, etc.
    \end{itemize}
    \textbf{Ergebnisse:}
    \begin{itemize}
        \item Aufzeigen des Einflusses auf die Flugleistungen
        \item Ermittlung des Optimums für die Flugleistung
    \end{itemize}
    }
\end{wpd}

% WPD 5100 

\clearpage
\begin{wpd}{AP 5100}{Diskussion der Ergebnisse}{1.0}{Lucas Schreer}{26.11.2018}{08.02.2019}{22.02.2019}{2 Woche}{Lucas Schreer}
    {
    \textbf{Ziele:}
    \begin{itemize}
        \item Festhalten der optimalen Parameter zur Erfüllung der Mission
        \item Bewertung der Ergebnisse im Hinblick auf Korrektheit und technischer Realisierbarkeit
        \item Empfehlungen für die optimale Auslegung eines Flugsystems für einen Steiglug auf \SI{10}{km} Höhe
    \end{itemize}
    \textbf{Input:}
    \begin{itemize}
        \item Ergebnisse aus AP 4200 und AP 4300
    \end{itemize}
    \textbf{Schnittstellen zu anderen APs:}
    \begin{itemize}
        \item AP 4000
    \end{itemize}
    \textbf{Aufgaben:}
    \begin{itemize}        
        \item Kritische Betrachtung der Ergebnisse und der gemachten Angaben 
        \item Auswertung der Ergebnisse im Hinblick auf die bestmöglichen Flugeigenschaften
    \end{itemize}
    \textbf{Ergebnisse:}
    \begin{itemize}
        \item Aussagen über eine bestmögliche Konstellation der Flugsystemparameter zum Erreichen einer Höhe von \SI{10}{km} oder sogar \SI{15}{km}
        \item Aussagen über die Realisierbarkeit 
        \item Wissen um die Abweichungen von der Realität und deren Einfluss 
        \item Ausblick auf zukünftige Entwicklungen
    \end{itemize}
    }
\end{wpd}

% WPD 5200

%\clearpage
%\begin{wpd}{AP 5200}{Ausblick}{1.0}{Lucas Schreer}{26.11.2019}{15.02.2019}{20.02.2019}{5 Tage}{Lucas Schreer}
%    {
%    \textbf{Ziele:}
%    \begin{itemize}
%        \item 
%    \end{itemize}
%    \textbf{Input:}
%    \begin{itemize}
%        \item Ergebnisse aus AP 5100 und AP 5300
%    \end{itemize}
%   	\textbf{Schnittstellen zu anderen APs:}
%  	\begin{itemize}
%        \item AP 5000
%	\end{itemize}
%    \textbf{Aufgaben:}
%    \begin{itemize}
%    	\item
%    \end{itemize}
%    \textbf{Ergebnisse:}
%    \begin{itemize}
%        \item Ausblick auf zukünftige Entwicklungen
%    \end{itemize}
%    }
%\end{wpd}

% WPD 6000

\clearpage
\begin{wpd}{AP 6000}{Dokumentation}{1.0}{Lucas Schreer}{26.11.2018}{04.12.2018}{22.02.2019}{11 Wochen}{Lucas Schreer}
    {
    \textbf{Ziele:}
    \begin{itemize}
        \item Schriftliche Dokumentation der Arbeit
    \end{itemize}
    \textbf{Input:}
    \begin{itemize}
        \item AP 1000
        \item AP 2000
        \item AP 3000
        \item AP 4000
        \item AP 5000
    \end{itemize}
    \textbf{Aufgaben:}
    \begin{itemize}
        \item Einarbeitung in Zeichensatzprogramme, wie \LaTeX, {\fontfamily{cmr}Ti{\em k}Z}, \emph{PGF} und \emph{Gnuplot}
        \item Schriftliche Ausarbeitung der Arbeit
    \end{itemize}
    \textbf{Ergebnisse:}
    \begin{itemize}
        \item Bachelorarbeit
    \end{itemize}
    %\setcounter{wpdCurrentPage}{\value{wpdCurrentPage}-1}
    %\refstepcounter{wpdCurrentPage}\label{cnt:wpdTotalPages}
    }
\end{wpd}