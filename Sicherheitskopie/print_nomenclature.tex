% ******************************************************************************
% Ausgabe der Nomenklatur
%
% Fuege das Kapitel als Nomenklatur hinzu.
\addcontentsline{toc}{chapter}{Nomenklatur}
\chapter*{Nomenklatur}
% ******************************************************************************
%
% Glossareintraege
% ******************************************************************************
\section*{Glossar}
\paragraph{Single Input Single Output}
Eingrößensystem
\paragraph{Software Deployment}
Das Software Deployment umfasst die gesamten
Entwicklungsaktivitäten die den Einsatz der Software ermöglichen.
\paragraph{\texttt{V}i \texttt{IM}proved}
Einer der essenziell wichtigsten Texteditoren des Universums.
\texttt{V}i \texttt{IM}prove dist eine Weiterentwicklung des Texteditors
vi und funktioniert wie der vi-Editor im Textmodus auf jedem Terminal!
% ******************************************************************************
%
% Abkuerzungen
% ******************************************************************************
\section*{Akrynome}
\begin{longtable}{lp{13cm}}
	CVT & Continuously Variable Transmission \\
	ESC & Electronic Speed Control (Motorregler)\\
	PWM & Pulsweitenmodulation\\
	TOC & Top Of Climb\\
	UAV & Unmanned Aerial Vehicle\\
	VTOL & Vertical Take Off and Landing
\end{longtable}
% ******************************************************************************
%
% Lateinische Bezeichnungen
% ******************************************************************************
\section*{Lateinische Bezeichnungen}
\begin{longtable}{lp{2.5cm}p{10.5cm}}
	\textbf{Notation} & \textbf{Einheit} & \textbf{Beschreibung}\\
	\ensuremath{a}	& \si{m/s}		& Schallgeschwindigkeit \\
	\ensuremath{A}	& \si{N}		& aerodyn. Auftriebskraft \\
	\ensuremath{c}	& -				& Beiwert \\
	\ensuremath{C}	& \si{As}		& Kapazität \\
	\ensuremath{\dot{C}} & \si{1/s}	& Entladerate (C-Rate) \\
	\ensuremath{E}	& -				& Gleitzahl \\
	\ensuremath{F}	& \si{m^2}		& Fläche \\
	\ensuremath{g} 	& \si{m/s^2} 	& Erdbeschleunigung \\
	\ensuremath{H}	& \si{m}		& Höhe \\
	\ensuremath{\dot{H}} & \si{m/s}	& Höhenänderung (Steiggeschwindigkeit)\\
	\ensuremath{i}	& -				& Übersetzungsverhältnis \\
	\ensuremath{i}	& - 			& Zählervariable \\
	\ensuremath{I}	& \si{A}		& elektr. Strom \\
	\ensuremath{K}	& -				& Konstante \\
	\ensuremath{m}	& \si{kg}		& Masse \\
	\ensuremath{M}	& \si{N/m}		& Drehmoment \\
	\ensuremath{Ma}	& -				& Machzahl \\
	\ensuremath{n}	& -				& Anzahl \\
	\ensuremath{p}	& \si{Pa}		& Druck \\
	\ensuremath{P}	& \si{W}		& Leistung \\
	\ensuremath{r}	& \si{m}		& Radius \\
	\ensuremath{R}	& \si{J/(kg.K)}	& Gaskonstante der Luft \\
	\ensuremath{R}	& \si{\ohm}		& elektr. Widerstand \\
	\ensuremath{S}	& \si{N}		& Schubkraft \\
	\ensuremath{T}	& \si{K}		& Temperatur\\
	\ensuremath{u}	& \si{m/s}		& Windgeschwindigkeit in x-Richtung \\
	\ensuremath{U}	& \si{V}		& elektr. Spannung \\
	\ensuremath{V}	& \si{m/s}		& Fluggeschwindigkeit \\
	\ensuremath{\dot{V}} & \si{m/s^2} & Ableitung der Geschwindigkeit (Beschleunigung) \\
	\ensuremath{w}	& \si{m/s}		& Windgeschwindigkeit in z-Richtung \\
	\ensuremath{W}	& \si{N}		& aerodyn. Widerstandskraft \\
	\ensuremath{x,X}& -				& in Längsrichtung des Fluggeräts \\
	\ensuremath{y,Y}& -				& in Seitenrichtung des Fluggeräts \\
	\ensuremath{z,Z}& -				& in Höhenrichtung des Fluggeräts
	
	
%	$\MM{A}, \MM{P}, \MM{Q}$ & - &
%		Positiv definite Matrizen der linearen \textsc{Lyapunov}
%		Matrix-Gleichung\\
%	$F$ & \si{kg.m/s^2} & Kraft\\
%	$\vv{x}$ & - &
%		Zustandsvektor eines Zustandsraummodells, der Form
%		\mbox{
%		    $\dot{\vec{x}} = \vec{f}(\vec{x}) + \vec{g}(\vec{x}) u
%		    \text{,}\quad
%			y = h(\vec{x})$
%		}\\
%	$y$ & - & Ausgangsgröße
\end{longtable}
% ******************************************************************************
%
% Griechische Bezeichnungen
% ******************************************************************************
\section*{Griechische Bezeichnungen}
\begin{longtable}{lp{2.5cm}p{10.5cm}}
	\textbf{Notation} & \textbf{Einheit} & \textbf{Beschreibung}\\
	\ensuremath{\alpha}	& \si{^\circ}		& Anstellwinkel \\
	\ensuremath{\gamma}	& \si{^\circ}		& Bahnneigungswinkel (Steigwinkel)\\
	\ensuremath{\dot{\gamma}}& \si{^\circ /s}& Ableitung des Bahnneigungswinkel (Änderungsrate) \\
	\ensuremath{\eta}	& \si{\%}			& Wirkungsgrad \\
	\ensuremath{\Theta} & \si{^\circ}		& Neigungswinkel \\
	\ensuremath{\kappa}	& -					& Adiabatenexponent \\
	\ensuremath{\rho}	& \si{kg/m^3}		& Luftdichte \\
	\ensuremath{\sigma}	& \si{^\circ}		& Schubeinstellwinkel\\
	\ensuremath{\omega}	& \si{J/kg}			& Energiedichte \\
	\ensuremath{\Omega}	& \si{1/s} o. \si{RPM}	& Drehzahl
	
		
	
	
%	$\beta, \gamma$ & - & Funktionen bestimmter monotoner Klassen\\
%	$\vv{\nu}_{ad}$ & - & Adaptiver Anteil der Pseudosteuergröße\\
%	$\overline{\sigma}, \underline{\sigma}$ & - &
%		maximaler und minimaler Singulärwert\\
%	$\vv{\xi}$ & - & Teilzustandsvektor der Zustände des transformierten
%		Systems, auch als externe Dynamik bezeichnet
\end{longtable}
% ******************************************************************************
%
% Indizes
% ******************************************************************************
\section*{Indizes}
\begin{longtable}{lp{13cm}}
	\textbf{Notation} & \textbf{Beschreibung}\\
	$i, j, k, l$ & Bezeichnen mit dem Index eine Komponente eines Tensors,
		während bei doppelten Index eine Summation der Komponenten nach
		der textsc{Einstein}schen Summenkonvention erfolgt.\\
    $\alpha , \beta , \gamma$ & Bezeichnen mit dem Index eine Komponente eines
    	Tensors, auch wenn ein doppelter Index vorliegt.
\end{longtable}
% ******************************************************************************
%
% Hochgestellte Indizes
% ******************************************************************************
\section*{Hochgestellte Indizes}
\begin{longtable}{lp{13cm}}
	\textbf{Notation} & \textbf{Beschreibung}\\
	$(\Delta )$ & Beschreibt die jeweilige Größe im Hauptachsensystem\\
	$(r)$ & Bezeichnet den schnellen Anteil der Druck-Scher-Korrelation
	        (engl.: rapid $=r$)\\
	$(s)$ & Bezeichnet den langsamen Anteil der Druck-Scher-Korrelation
        (engl.: slow $=s$)\\
	$\left(\widetilde{b}_{ij}\right)$ & Beschreibt Größen, die zu dem
		gemittelten Anisotropietensor der
		\textsc{Reynolds}-Spannungungen $\widetilde{\MM{b}}_{ij}$
		bestimmt wurden.
\end{longtable}
%
% Korrektur des Tabellenzaehlers, da für die Darstellung der
% Symbolverzeichnisses Tabellen verwendet werden, die jedoch nicht zu dem Inhalt
% der Arbeit gehoeren.
\addtocounter{table}{-5}
% ******************************************************************************
