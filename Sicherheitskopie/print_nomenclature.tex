% ******************************************************************************
% Ausgabe der Nomenklatur
%
% Fuege das Kapitel als Nomenklatur hinzu.
\addcontentsline{toc}{chapter}{Nomenklatur}
\chapter*{Nomenklatur}
% ******************************************************************************
%
% Glossareintraege
% ******************************************************************************
\section*{Glossar}
\paragraph{Single Input Single Output}
Eingrößensystem
\paragraph{Software Deployment}
Das Software Deployment umfasst die gesamten
Entwicklungsaktivitäten die den Einsatz der Software ermöglichen.
\paragraph{\texttt{V}i \texttt{IM}proved}
Einer der essenziell wichtigsten Texteditoren des Universums.
\texttt{V}i \texttt{IM}prove dist eine Weiterentwicklung des Texteditors
vi und funktioniert wie der vi-Editor im Textmodus auf jedem Terminal!
% ******************************************************************************
%
% Abkuerzungen
% ******************************************************************************
\section*{Akrynome}
\begin{longtable}{lp{13cm}}
	FAA & Federal Aviation Administration\\
	IFF & Institut für Flugführung\\
	NASA & National Aeronautics and Space Administration\\
	SISO & Single Input Single Output\\
	VIM & \texttt{V}i \texttt{IM}proved
\end{longtable}
% ******************************************************************************
%
% Lateinische Bezeichnungen
% ******************************************************************************
\section*{Lateinische Bezeichnungen}
\begin{longtable}{lp{2.5cm}p{10.5cm}}
	\textbf{Notation} & \textbf{Einheit} & \textbf{Beschreibung}\\
	$\MM{A}, \MM{P}, \MM{Q}$ & - &
		Positiv definite Matrizen der linearen \textsc{Lyapunov}
		Matrix-Gleichung\\
	$F$ & \si{kg.m/s^2} & Kraft\\
	$\vv{x}$ & - &
		Zustandsvektor eines Zustandsraummodells, der Form
		\mbox{
		    $\dot{\vec{x}} = \vec{f}(\vec{x}) + \vec{g}(\vec{x}) u
		    \text{,}\quad
			y = h(\vec{x})$
		}\\
	$y$ & - & Ausgangsgröße
\end{longtable}
% ******************************************************************************
%
% Griechische Bezeichnungen
% ******************************************************************************
\section*{Griechische Bezeichnungen}
\begin{longtable}{lp{2.5cm}p{10.5cm}}
	\textbf{Notation} & \textbf{Einheit} & \textbf{Beschreibung}\\
	$\beta, \gamma$ & - & Funktionen bestimmter monotoner Klassen\\
	$\vv{\nu}_{ad}$ & - & Adaptiver Anteil der Pseudosteuergröße\\
	$\overline{\sigma}, \underline{\sigma}$ & - &
		maximaler und minimaler Singulärwert\\
	$\vv{\xi}$ & - & Teilzustandsvektor der Zustände des transformierten
		Systems, auch als externe Dynamik bezeichnet
\end{longtable}
% ******************************************************************************
%
% Indizes
% ******************************************************************************
\section*{Indizes}
\begin{longtable}{lp{13cm}}
	\textbf{Notation} & \textbf{Beschreibung}\\
	$i, j, k, l$ & Bezeichnen mit dem Index eine Komponente eines Tensors,
		während bei doppelten Index eine Summation der Komponenten nach
		der textsc{Einstein}schen Summenkonvention erfolgt.\\
    $\alpha , \beta , \gamma$ & Bezeichnen mit dem Index eine Komponente eines
    	Tensors, auch wenn ein doppelter Index vorliegt.
\end{longtable}
% ******************************************************************************
%
% Hochgestellte Indizes
% ******************************************************************************
\section*{Hochgestellte Indizes}
\begin{longtable}{lp{13cm}}
	\textbf{Notation} & \textbf{Beschreibung}\\
	$(\Delta )$ & Beschreibt die jeweilige Größe im Hauptachsensystem\\
	$(r)$ & Bezeichnet den schnellen Anteil der Druck-Scher-Korrelation
	        (engl.: rapid $=r$)\\
	$(s)$ & Bezeichnet den langsamen Anteil der Druck-Scher-Korrelation
        (engl.: slow $=s$)\\
	$\left(\widetilde{b}_{ij}\right)$ & Beschreibt Größen, die zu dem
		gemittelten Anisotropietensor der
		\textsc{Reynolds}-Spannungungen $\widetilde{\MM{b}}_{ij}$
		bestimmt wurden.
\end{longtable}
%
% Korrektur des Tabellenzaehlers, da für die Darstellung der
% Symbolverzeichnisses Tabellen verwendet werden, die jedoch nicht zu dem Inhalt
% der Arbeit gehoeren.
\addtocounter{table}{-5}
% ******************************************************************************
