\chapter{Theoretische Grundlagen}
\label{chap:theoretical_principles}
\section{Einsatz von VIM}
\label{sec:einsatz_von_vim}
Einer der essenziell wichtigsten Texteditoren des Universums.  \texttt{V}i
\texttt{IM}prove dist eine Weiterentwicklung des Texteditors vi und funktioniert
wie der vi-Editor im Textmodus auf jedem Terminal! Die Bedienung erfolgt dann
üblicherweise über die Tastatur, eine Maus wird zwar auf vielen Terminals
unterstützt, ihre Verwendung ist aber limitiert \cite{Mustermann2017}.

Es sei
\begin{equation} e^{(r)} (t) = -c_0e (t) - c_1 \dot{e} (t) - \dots - c_{r-1}
	e^{(r-1)} (t) \eqend{,}
\label{eq:Fehlerdynamik_siso}
\end{equation}
respektive
\begin{equation}
	e^{(r)} (t) + \MM{c}^T\vv{\chi} = 0
	\eqend{.}
\label{eq:Fehlerdynamik_siso_matrix}
\end{equation}
\blindtext[2]
\section{Section 2}
\label{sec:section_one}
\blindtext[1]
\blindlist{itemize}[2]
\blinddescription
\blindmathtrue
\blindmathfalse

\begin{figure}[!ht]
	\centering
	\missingfigure[figwidth=0.75\textwidth]{Testing a long text string}
	\label{fig:dummy}
	\caption{Dummy figure}
\end{figure}
\blindmathpaper
