\chapter{Zusammenfassung und Ausblick}

\section{Zusammenfassung}
In dieser Arbeit wurde ein Programm zur flugmechanischen Untersuchung von elektrischen, propellergetriebenen Fluggeräten zum effizienten Aufstieg in die untere Stratosphäre vorgestellt und genauer erläutert (vgl. Abschn. \ref{sec:aufbau_des_programms}). Im Anschluss an die Darlegung der technischen Grenzen und der Genauigkeit der Modelle(vgl. Abschn. \ref{sec:vernachlaessigungen_vereinfachungen}) erfolgte die Validierung des Programms anhand eines realen Steigfluges eines Quadrocopters auf über \SI{10}{km} Höhe (vgl. Kap. \ref{chap:nachbildung_des_quadrocopter}). Es zeigte sich, dass die Flugleistungen des Quadrocopters mit den Berechnungen des Programms gut übereinstimmen (vgl. Abschn. \ref{sec:ergebnisse_quadrocopter}). \\
Die erste Optimierung des Fluggerätes stellte die grundsätzliche Frage dar, welche Art von Fluggerät, Flächenflugzeug oder Multicopter, besser den Ansprüchen eines effizienten Aufstiegs und Randbedingungen des Projektes AEROMET\_UAV gerecht wird (vgl. Kap. \ref{chap:vergleich}). Hierzu wurde das Flächenflugzeug im möglichen Rahmen des Modells optimiert und die Ergebnisse anschließend mit den Flugleistungen des Quadrocopters aus Kap. \ref{chap:nachbildung_des_quadrocopters} verglichen. Der Multicopter erwies sich für die Anwendung innerhalb des AEROMET\_UAV Projektes als besser geeignet (vgl. Abschn. \ref{sec:ergebnisse_vergleich}). Infolge dessen wurden im Anschluss Parameter des Multicopters optimiert. \\
Für die Optimierung wurde zuerst eine Multicopter-Referenzkonfiguration definiert (vgl. Abschn. \ref{sec:multicopter_referenzkonfig}), an der die Einflüsse verschiedener Parameter deutlich gemacht wurden. Die Optimierung befasste sich mit dem Einfluss von Parametern wie dem Widerstandsbeiwert (vgl. Abschn. \ref{subsec:widerstandseinfluss}), der Batteriespannung (vgl. Abschn. \ref{subsec:einfluss_n_bat}), dem Motorreglerwirkungsgrad (vgl. Abschn. \ref{subsec:einfluss_eta_pwm}) oder dem Einfluss des maximalen Motorstroms (vgl. Abschn. \ref{subsec:einfluss_imax}).\\
Eine wichtige Optimierung betraf die Konfiguration des Multicopters an sich. Hierzu wurde zum einen die optimale Batteriemasse in Abhängigkeit der Gesamtmasse (vgl. Abschn. \ref{subsec:massenverteilung}) untersucht sowie zum anderen die Größe des Multicopters (vgl. Abschn. \ref{subsec:groesse}) und die Anzahl der Propeller (vgl. Abschn. \ref{subsec:anz_prop}). 
Weiterhin wurde der Einsatz eines Verstellpropellers (vgl. Abschn. \ref{subsec:verstellprop}) und eines Getriebes (vgl. Abschn. \ref{subsec:getriebe}) abgewogen. Beide können nur unter idealen Voraussetzungen einen Leistungsgewinn erzielen.
Mit diesen Ergebnissen wurde ein optimale Lösung entwickelt (vgl. Abschn. \ref{subsec:vorueberlegung}). Die optimale Lösung demonstrierte gute Flugleistungen, mit denen ein Steigflug auf mehr als \SI{18000}{m} unter idealen Umgebungsbedingungen möglich ist (vgl. Abschn. \ref{subsec:ideale_rb}). Unter den Randbedingungen des AEROMET\_UAV-Projektes fallen die Ergebnisse schlechter aus, jedoch ist in einer leicht modifizierten Konfiguration zur optimalen Lösung immer noch eine Flughöhe von bis zu \SI{15000}{m} erreichbar. Diese Höhe könnte durch die Verwendung von aerodynamischen Gleitflächen am Multicopter erhöht werden, wenn damit ein antriebsloser Gleitflug im anschließenden Sinkflug möglich ist.


\section{Ausblick}
Der nächste Schritt ist die Validierung der in dieser Arbeit aufgestellten Ergebnisse. Das umfasst insbesondere Flugversuche und Flugleistungsmessungen wie z.B. in \cite{Ostler.2006} oder \cite{PCUP.2017}.
Außerdem ist eine Verfeinerung der Modelle, die dem Programm zugrunde gelegt wurden, anzustreben. Insbesondere das Modell des Motorreglers, der Motoren oder der Batterie benötigen eine Abhängigkeit von der Temperatur.
Ebenso ist das Modell des Flächenflugzeugs zu erweitern. Bisher ist nur die Leistungsgrenze berücksichtigt worden, nicht aber z.B. die Auftriebsgrenze oder die Festigkeitsgrenze innerhalb der Flugenveloppe. 
Ein weiterer wichtiger Punkt ist, dass allen Modellen bis auf das Modell des Motors eine Massenabhängigkeit fehlt. Ein Beispiel dafür ist die Erhöhung der Gleitleistung eines Flugzeuges oder eine Verringerung der Flügeldicke. Werden diese Parameter geändert, so zieht das zusätzlich eine Änderung der Flügelform oder Profildicke nach sich, die wiederum eine verstärkte Flügelstruktur und Flugzeugzelle und letztendlich eine Anpassung der Masse erfordern. Dieser Zusammenhang findet hier keine Anwendung. Es fehlen Funktionen und Datenbanken, um diese Abhängigkeit darzustellen.\\
Bei der bisherigen Untersuchung und Optimierung von Parametern von unbemannten Fluggeräten ist bisher nur der Einfluss einzelner Parameter unter Festhaltung aller übrigen Parameter betrachtet worden. An dieser Stelle fehlt eine globale Optimierung, die alle Variationen aller Parameter durchrechnet und am Ende das beste Ergebnis aller Parameterkombinationen präsentiert. Dies erhöht die Genauigkeit und berücksichtigt einen gegenseitigen Einfluss von zu untersuchenden Aspekten. Ein Modell hierzu wird z.B. in \cite{Magnussen.2015} vorgeschlagen. 