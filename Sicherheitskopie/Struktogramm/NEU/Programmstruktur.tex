\documentclass{article}
\usepackage{struktex}
\usepackage{geometry}
\usepackage[ngerman]{babel}
\geometry{a5paper, top=0mm, left=-0.5mm, right=5mm, bottom=10mm}



\begin{document}

\pagestyle{empty}

\vspace*{\fill}

\begin{struktogramm}(140,185)
\assign[1]{Fluggerät auswählen und Komponenten definieren(im Startskript)}
\assign[1]{Missions- und Umgebungsparameter festlegen (im Startskript)}
\assign[1]{Diskretisierungen festlegen}
\assign[1]{Aufruf des Hauptskripts: Leistungsberechnung starten}
\assign[1]{Initialisierung der Parameterberechnung}
\while[5]{F\"ur alle Höhenabschnitte}
	\assign[1]{H\"ohe, Dichte, Luftdruck Temperatur berechnen}
	\assign[1]{arithmetische Mittelwert berechnen}
	\assign[1]{Schub- und Leistungskennfeld anpassen}
	\assign[2]{Initialisierung der Leistungsberechnung}
	\while[5]{Für alle Bahnneigungswinkel}
		\assign[2]{\textbf{Leistungsberechnung}}
	\whileend
	\ifthenelse[10]{1}{5}{Fluggerät?}{Multicopter (1)}{Flächenflugzeug (0)}
		\assign[2]{Übergabe der zwischengespeicherten Leistungsparameter}
		\change
		\ifthenelse[10]{4}{1}{Sind die Werte NaN?}{nein}{ja}
			\while[5]{Solange Abbruchkriterium nicht erreicht}		
				\assign[2]{Finde den Index mit der geringsten verbrauchten Energiemenge}
				\ifthenelse[10]{1}{1}{Werte innerhalb Leistungsgrenzen?}{ja}{nein}
				\assign[2]{Verlasse Schleife}
				\change
				\assign[2]{Suche nächst kleinere Energiemenge}
				\ifend
			\whileend
			\assign[2]{Übergabe aller Leistungsparameter mit diesem Index}
			\change
			\assign[2]{Verwerfe alle Ergebnisse}
		\ifend
	\ifend
\whileend
\assign[2]{Ergebnisse der Leistungsparameter in Diagrammen speichern}
\assign[2]{Speichern der Diagramme in .pdf-Datei}
\end{struktogramm}

\end{document}