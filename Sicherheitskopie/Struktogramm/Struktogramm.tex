\documentclass{article}
\usepackage{struktex}
\usepackage{geometry}
\usepackage[ngerman]{babel}
\geometry{a5paper, top=0mm, left=-0.5mm, right=5mm, bottom=10mm}



\begin{document}

\pagestyle{empty}

\vspace*{\fill}

\begin{struktogramm}(140,185)
\assign[2]{Flugger\"at ausw\"ahlen (im Startskript)}
\assign[2]{Flugger\"atkomponenten definieren (im Startskript)}
\assign[2]{Missionsparameter festlegen (im Startskript)}
\assign[2]{Umgebungsparameter festlegen (im Startskript)}
\assign[2]{Aufruf des Hauptskripts: Leistungsberechnung starten}
\assign[2]{Initialisierung der Parameterberechnung}
\while[5]{F\"ur alle Höhenabschnitte}
	\assign[2]{H\"ohe, Dichte, Luftdruck Temperatur berechnen}
	\assign[2]{arithmetische Mittelwert berechnen}
	\assign[2]{Schub- und Leistungskennfeld anpassen}
	\ifthenelse[15]{1}{1}{Flugger\"at?}{Multicopter (1)}{Fl\"achenflugzeug (0)}
		\while[5]{Solange Abbruchkriterium nicht erreicht}
			\assign{Aerodynamik berechnen}
		\whileend
		\assign[2]{Schub berechnen}
		\change
		\assign[2]{Bahnneigungswinkel aus reziproker Gleitzahl berechnen}
		\assign[2]{Schub berechnen}
	\ifend
	\assign[2]{Schub auf Propeller verteilen}
	\ifthenelse[10]{1}{4}{Schub zu gro\ss{}?}{ja}{nein}
		\assign[2]{Ergebnis verwerfen (NaN)}
		\change
		\assign[2]{Drehzahl und Drehmoment aus Propellerkennfeld interpolieren}
		\assign[2]{Motorzustand berechnen}
		\assign[2]{Zustand der Motorregler berechnen}
		\assign[2]{Zustand der Batterie neu berechnen}
	\ifend
	\ifthenelse[10]{1}{1}{Werden Grenzen \"uberschritten?}{ja}{nein}
	\assign[2]{Ergebnis verwerfen (NaN)}
	\change
	\assign[2]{Ergebnis beibehalten}
	\ifend
\whileend
\assign[2]{Ergebnisse für Restladung, Drehzahl, Motorstrom und -spannung, PWM in Diagramme zeichnen}
\assign[2]{Speichern der Diagramme als .jpg - Bilder}
\end{struktogramm}

\end{document}