\chapter*{Übersicht}
In dieser Arbeit wird eine Leistungsuntersuchung an lektrisch, propellergetriebenen Fluggeräten durchgeführt. Dies umfasst sowohl Multicopter als auch Flächenflugzeuge. Das Ziel ist es eine Höhe von \SI{10}{km} bis \SI{15}{km} zu erreichen. Der Hintergrund ist der angedachte Austausch von Wetterballonen für die Atmosphärenmessung durch solche Fluggeräte, da diese viele Nachteile besitzen, die die Fluggeräte nicht haben. Im März des Jahres 2018 zeigte ein Quadrocopterflug, dass es möglich ist eine Höhe von mehr als \SI{10000}{m} zu erreichen. 
Nach einem kurzen Standpunkt der Technik folgt die Beschreibung der Flugleistungsberechnung innerhalb des dafür vorgesehenen Programms. Dies umfasst auch die Grenzen und Einschränkungen der verwendeten Modelle. Eine Überprüfung und Validierung des Programms erfolgt mit dem Abgleich eines realen Steigfluges auf \SI{12600}{m} mit einem Quadrocopter und des im Programm errechneten Flugleistungen. Dabei reproduziert das Programm die Flugleistungen akkurat. Es gibt allerdings auch gewisse Abweichungen bzgl. der Motorregler. \\
Die eigentliche Parameteruntersuchung beginnt mit dem Vergleich der Flugleistungen von einem Multicopter mit einem äquivalenten Flächenflugzeug. Der Multicopter weist im Gegensatz zu einem Flächenflugzeug entscheidende Vorteile für diese Mission auf und noch mehr Potential. Aus diesem Grund wird er im weiteren Verlauf genauer betrachtet. Das umfasst vor allem den Motor, die Propeller, die Batterie sowie den Anteil der Batteriemasse am Gesamtgewicht. Zum Schluss wird noch der Leistungsgewinn durch den Einsatz von einem Verstellpropeller und einem Getriebe untersucht. 



